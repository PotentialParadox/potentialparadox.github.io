\chapter{Spectroscopic Analysis of PPV\(_3\)-NO\(_2\)}

\section{Introduction}
NEXMD is an efficient program for the simulation of photoinduced
dynamics of extended conjugated molecular systems involving manifolds
of coupled electronic excited states over timescales extending up to
tens of picoseconds. \cite{sifain2018photoexcited, Bjorgaard2015,
  case2020a, tretiak02_densit_matrix_analy_simul_elect,
  malone2020nexmd} It includes solvent effects using implicit
models. These implicit solvents provide insight into the electrostatic
forces. Still, explicit solvents should provide additional information
about the quantum or steric effects, allowing the simulation of
electron transfer processes due to the stabilization of charge
separation in the excited state.\cite{woo2005solvent}  Many multichromophoric molecular
systems are soluble in polar solvents such as water. Such simulations
could provide sought-after insight into the effects of charged side
groups on the structural sampling, structural rearrangement, and
transition density redistribution during electronic relaxation.

Adequate sampling of the solvent and solute configuration space,
including hundreds to thousands of solvent molecules, currently
cannot be achieved by QM calculations alone due to the infeasible
computational cost of such extensive
systems. \cite{barbatti2011nonadiabatic} QM/MM methods exist to allow
the computationally expensive QM calculations to be performed only on
the area of interest while propagating the rest of the system with
cheaper classical dynamics.  The SANDER program in the AMBER
molecular dynamics package performs classical molecular dynamics
using periodic boundary conditions with high optimization.  SANDER
with QM/MM performs well with tens of thousands of MM atoms and
currently calculates QM/MM with semi-empirical Hamiltonians and DFT;
however, previous versions did not provide options for excited-state
MD.

In this work, we redirect AMBER's SANDER package from its usual
semi-empirical QM package, SQM, to a modified NEXMD library.
SANDER linked to the NEXMD library performs adiabatic dynamics at
ground-state and CIS excited-state potential energy surfaces on
~100 atoms at QM and 1000s at MM.  We apply our method to an
analysis of a three-ring para-phenylene vinylene oligomer
\((PPV_{3}-NO_{2})\).  We look at how explicit solvents affect \(PPV_{3}-NO_{2}\)'s
absorption and emission spectra as well as its geometric structure.

\section{Simulation Methods}

\noindent
\begin{minipage}[c]{\textwidth}
  \centering
  \includegraphics[width=5in]{../Paper1/Images/trajectory_diagram/trajectories.pdf}
  \captionof{figure}{Description of the adiabatic simulation}
  \label{fig:adiabaticDynamics}
\end{minipage}\bigskip

Figure \ref{fig:adiabaticDynamics} visually describes the layout of our simulations performed in vacuum, methanol, chloroform, and carbon tetrachloride.
   First, we performed a 320 ps molecular dynamics simulation at the ground state (S0) at 300K (NVT) using the General AMBER Force Field and periodic boundary conditions, as guided from previous convergence studies. \cite{silva2010benchmark}
We set the classical time-step to 0.5 fs and use a Langevin thermostat with friction constant 2 ps\(^{-1}\) throughout all simulations. The choices for these parameters were based on work aimed at optimizing these choices in NEXMD. \cite{nelson2012nonadiabatic}
   128 evenly-spaced snapshots were then collected from the MM simulation and used as initial conditions for 128 individual 10 ps simulations using the semi-empirical AM1 QM/MM ground state Hamiltonian that has been shown to provide reasonable accuracy for the computational cost. \cite{silva2010benchmark} We use the data collected from these trajectories for the absorption calculation.
   The resulting geometries of these ground-state QM/MM simulations were then used as the initial geometrical configuration of a final excited-state simulation. Each configuration was vertically excited to the lowest excited state with no regard to coupling or energy conservation. We propagate these excited-state QM trajectories using configuration interaction singles (CIS) with the AM1 Hamiltonian, 
   The CIS calculations include the first 9 excited states of the PPV\(_3\)-NO\(_2\) molecule and the stated number of solvent molecules closest to the central phenyl.
   We treat all other solvent molecules using classical dynamics.
   We restrict the QM solvents from drifting away from the solute and the other MM solvents from drifting closer than these QM solvents.

\noindent
	  \begin{multiFigure} 
	    \addFigure{0.45}{../Paper1/Images/ppvno2.png}
	    \addFigure{0.45}{../Paper1/Images/vacuum-td.png}
	    \captionof{figure}{A) Diagram of PPV\(_3\)-NO\(_2\) showing the bonds of interest. B) Transition density from ground to excited state.}
	    \label{fig:PPV3NO2}
	  \end{multiFigure}
\bigskip

As shown in Figure \ref{fig:PPV3NO2} B, the S0-S1 transition density resides towards the center of the molecule on the vinyl groups nearest to the central phenyl group.
Charge movements on solvents far from this concentration of density should cause negligible effects on transition characteristics.
To maximize the QM/MM calculation utility, we only include the solute and the solvent molecules nearest to this central phenyl group in the QM calculations.
To prevent these solvents from drifting during the trajectories, we implement a simple harmonic restraint using AMBER's NMR restraints with a force constant of 200 KCal/mol \AA\(^{-2}\).
We select the N solvent molecules based on the minimum atom-atom distance between the solvent and the central phenyl group.
We then restrain these solvents using the harmonic constraint on the distance from the center of geometry of the solvent to the center of geometry of the central phenyl group.
We also use the same restraint to prevent the solvent molecules not included in the QM calculation from getting closer to the central phenyl than the QM solvents, effectively making a spherical shield around the solute's central phenyl group.
Since the distance between the center of geometries and the closest atoms are not necessarily equal, solvent atoms could initially be pushed either inside or outside this shield during the transition from MM to QM.
However, this push only occurs during the initial equilibration of the QM ground-state calculations excluded from any analysis.
Once the QM calculations begin, these constraints persist throughout all further calculations.
For our CCl\(_4\) simulations with the 5 nearest solvents included (CCl\(_4\)-5QM), the restriction barrier had an average radius of 6.28 \AA\ from the origin at the center of the central phenyl group.

\section{Results}

\subsection{Spectra}

Energies, coordinates, and dipole information are acquired every ten steps.
Equilibration times from either MM to QM state or from S0 to S1 range from 2-4 ps.
As such, we exclude the first four ps of each trajectory in calculating the spectra in data analysis for the absorption and emission analysis. 

The solvent's polarizability should affect the transition dipole moments and thus the corresponding spectra.  This effect has been shown experimentally. \cite{marcus1956electrostatic,martin1998hydrolysis,Park2013,LeDroumaguet2005}.
Such shifts are referred to as solvatochromic shifts.
Previous studies have analyzed solvatochromic shifts in conjugated substituted PPV\(_3\)-NO\(_2\) molecules with the NEXMD program in implicit solvents.\cite{Santhanamoorthi2009}
Results with NEXMD by TD-AM1 were redshifted from the experimental results, while single-point calculations using TD-CAM-B3LYP provided by G09 in the same implicit solvent were blue shifted.
Other NEXMD computations have shown comparable redshifts in spectra of similar molecules in implicit solvents compared to experiment. \cite{Bjorgaard2015}
We performed similar calculations in this paper; however, in explicit solvent.
We compare the results to those presented in implicit solvent.

We collect the vertical excitation dipoles and oscillator strengths between the ground state S0 and S1 every five fs during each trajectory's steady-state to produce the absorption/emission spectra of PPV\(_3\)-NO\(_2\).
We sum over excitation states averaged over the geometries and broaden the spectra using a Gaussian bin function with FWHM=0.16 eV corresponding to a 100 fs FWHM laser excitation.
We normalized it such that the maximum absorption is 1. 

\noindent
\begin{multiFigure} 
  \addFigure{0.45}{../Paper1/Images/spectra_abs_compared.png}
  \addFigure{0.45}{../Paper1/Images/spectra_flu_compared.png}
  \captionof{figure}[Fluorescence and absorption spectra in various solvents]{PPV\(_3\)-NO\(_2\) A) absorption and B) fluorescence spectra in various solvents with 20 included in QM region.}
  \label{fig:spectrasolvents}
\end{multiFigure}\bigskip

\noindent
\begin{multiFigure} 
  \addFigure{0.45}{../Paper1/Images/nquant_abs_comparison.png}
  \addFigure{0.45}{../Paper1/Images/nquant_flu_comparison.png}
  \captionof{figure}[Fluorescence and absorption spectra by number of quantum solvents]{PPV\(_3\)-NO\(_2\) A) absorption and B) fluorescence spectra in CCL\(_4\) with varying number of QM solvents.}
  \label{fig:spectranquant}
\end{multiFigure}\bigskip

Figure \ref{fig:spectrasolvents} A presents the absorption spectra for PPV\(_3\)-NO\(_2\) in select solvents and vacuum.
The shown absorption has contributions from the nine lowest energy excited states, though the S1 state is the primary contributor to the spectra.
We found the number of solvent molecules included in the QM region caused only minor deviations in the spectra, with the largest variance (0.02~eV) occurring between the 20QM and MM carbon tetrachloride systems, as seen in Figure \ref{fig:spectrasolvents}. As such, we only present results from trajectories with 20 QM solvents.
The broad absorption and fluorescence bands are common features of conjugated materials due to the large geometric relaxation in the lowest excited state. \cite{bredas2009molecular}
All solvent results are redshifted from those in vacuum matching findings in previous works.\cite{Park2013}
In Figure \ref{fig:spectranquant}, the absorbance within methanol and chloroform were very similar, with a peak shift from vacuum of -0.04 eV.
Within carbon tetrachloride, the magnitude of this shift increases slightly more to-0.06 eV. 

Aligning with previously reported results, the fluorescence calculations found in Figure \ref{fig:spectranquant} show an overall more intense redshift from vacuum, along with a more significant dispersion among the solvents. \cite{Park2013}
The smallest shift, at -0.06 eV, occurs in carbon tetrachloride, while the largest, at -0.12 eV, occurs in methanol.
Previous works have demonstrated that the energy levels of PPV\(_3\)-NO\(_2\) are further stabilized by more polar solvents, a feature clearly seen by our results.\cite{Park2013,woo2005solvent}

\subsection{Potential Energy Relaxation}

\noindent
\begin{minipage}[c]{\textwidth}
  \centering
  \includegraphics[width=5in]{../Paper1/Images/energies.png}
  \captionof{figure}[Potential energy relaxation during adiabatic dynamics]{Potential energ for states S0 and S1 for PPV\(_3\)NO\(_2\) with 20 solvent molecules included in the QM region as the system relaxes on the S1 PES. Energies are reported as differences from the ground state at t = 0.}
  \label{fig:energiesAdiabatic}
\end{minipage}\bigskip

The peaks of absorption and the fluorescence properties should correspond strongly with the difference between the ground state (S0) and the first excited state (S1) energies at the point of transistion.
Figure \ref{fig:energiesAdiabatic} shows the energies for states S0 and S1 averaged over 128 trajectories for PPV\(_3\)-NO\(_2\) in the various solvents with 20 solvent molecules included in the QM calculations.
During the first four ps in the chart, the system runs on the ground-state S0, and the S0 energies stay near the minimum with small oscillations caused by temperature.
At time t = 0, the system instantaneously hops to the S1 potential energy surface.
The average energy difference of around 2.95 eV at t = 0 corresponds reasonably well with whats seen in the absorption spectrums' peaks where little difference is seen between the solvents. During the first 10-20 fs of this calculation, the ground state energy rapidly increases, suggesting an ultra-fast geometric relaxation process occurs. The molecule then seems to experiences a slower, ~2 ps, relaxation process.
As expected the final energy values at t = 10 ps for S0 and S1 are higher and lower, respectively, than they were when t < 0.
Unexpectedly, PPV\(_3\)-NO\(_2\) in methanol seems to experience a minimum in s1 energy at around t = 2 ps then steadily increase until around t = 8 ps. During this time, the S0 energy also increases, alowing the energy difference found in methanol to remain lower than that found in other solvents which aligns well with the red shift found in the fluorescence calculations. This may be due to solvent rearrangement, however further inspection did not lead to anything conclusive.

\subsection{Bond Length Alternation}
    In \(\pi\) conjugated organic compounds, the electrons unevenly distribute across the carbon-carbon \(\pi\) bonds, causing a dimerization or alternation of single-like carbon bonds and double-like carbon bonds.
    When promoted from the highest occupied molecular orbital (HOMO) to the lowest occupied molecular orbital (LOMO), the bonding-antibonding pattern switches, and the electron densities correspondingly adjust.
    As a result, the single-like carbons become more double-like and vice versa. \cite{bredas1999excited}
    This structural difference between the excited-state and ground-state of PPV3-molecules is presented clearly by distortions in the C=C and C-C bonds found in the vinylene segment.\cite{tretiak02_densit_matrix_analy_simul_elect, karabunarliev2000rigorous, karabunarliev2000adiabatic, nelson2014nonadiabatic}
    These distortions can be measure by bond length alternation (BLA)
    \begin{equation}
      \frac{d_i + d_e}{2} - d_c,
    \end{equation}
    where \(d_i\) and \(d_e\) are the interior and exterior bonds, and \(d_c\) is the central bond.
    This value represents the differences between the double and single bonds of the vinylene sections.
    The BLA is a descriptor for \(\pi\) bond distributions. \cite{tretiak2002conformational}
    For this system, we analyze the BLAs of two separate bond sets, the bonds d1-3 (near-set) and d4-6 (far-set) seen in Figure \ref{fig:PPV3NO2} A. 

    \noindent
    \begin{minipage}[c]{\textwidth} 
      \centering
      \includegraphics[width=6in]{../Paper1/Images/bla.png}
      \captionof{figure}[BLA of bonds during adiabatic dynamics]{BLA of bonds d1-3 (left) and d4-6 (right) during the last 4 ps of the S0 dynamics and 10 ps of the S1 dynamics.
        QM energy and force calculations include the 20 solvents nearest the central ring.}
      \label{fig:bla_adiabatic}
    \end{minipage}\bigskip


    \begin{table}[H]
      \caption[Adiabatic bond length alternation]{Bond Length Alternation (BLA) summary for PPV\(_3\)-NO\(_2\) in various solvents with 20 solvents included in the QM region.} \label{tab:adiabaticBLA}
      % \begin{center}
      \begin{tabularx}{\textwidth}{XXXXXXXXX}\hline
        Molecule   & d\(_1\)  & d\(_2\) & d\(_3\) & BLA\(_{\textbf{near}}\) & d\(_1\)  & d\(_2\) & d\(_3\) & BLA\(_{\textbf{far}}\)\\\hline
        Vacuum     & 1.429     & 1.375    & 1.418    & 0.049              & 1.441     & 1.365    & 1.427   & 0.069\\
        CCl\(_4\)  & 1.427     & 1.376    & 1.417    & 0.046              & 1.441     & 1.365    & 1.426   & 0.068\\
        CH\(_3\)OH & 1.422     & 1.382    & 1.415    & 0.037              & 1.444     & 1.362    & 1.431   & 0.076\\
        CHCl\(_3\) & 1.423     & 1.380    & 1.415    & 0.039              & 1.443     & 1.365    & 1.429   & 0.074\\\hline
      \end{tabularx}
    \end{table}

    Figure \ref{fig:bla_adiabatic} shows the bond length alternation of
    PPV\(_3\)-NO\(_2\) in various solvents during the last 4 ps of the S0 trajectories and the 10 ps of the S1 trajectories. At time t = 0, the systems instantaneously transition from S0 to S1.
    Within the first couple hundred femtoseconds after the
    excitation to S1, the central bonds (d2 and d5) expand, while the interior (d3 and d6) and
    exterior bonds (d1 and d4)  contract.  This bond restructuring is amplified by
    proximity to the nitro group for all solvent environments, where we
    see an average drop of 0.07 Å in sets d1-3 compared to a 0.04 Å drop
    found in sets d4-6.  This amplification's strength depends on the
    solvent environment where the BLA difference between the far and near
    sets PPV\(_3\)-NO\(_2\) in methanol, 0.034 Å, surpasses that found in carbon
    tetrachloride, 0.022 Å.

    Table \ref{tab:adiabaticBLA} presents further details of the S1 BLA simulation.
    The information presented is averaged over time after relaxation across all trajectories.
    In all cases, the exterior bonds become slightly longer than the interior bond.
    The BLAs from the near and far sets split.
    In the near set, the S0 and S1 BLA converge to 0.1091 Å and 0.00453 Å, respectively.
    In the far set, these numbers are 0.1103 Å and 0.0697 Å, respectively.
    The smaller bond length spread in the near set and the lower BLA suggest more delocalization on those bonds than in the far set. 
    The number of solvents included in the QM calculations had little effect on the convergence of the distances or BLA.
    PPV\(_3\)-NO\(_2\) had similar ground state BLAs of around 0.11 Å for both near and far sets matching results on similar systems regardless of the solvent. \cite{nelson2011nonadiabatic}
    The S1 BLAs varied between the solvents and the distance away from the nitro group.
    PPV\(_3\)-NO\(_2\) presents the smallest near set S1 BLA and largest far set S1 BLA when in methanol in comparison to the other tested solvents.
In carbon tetracholoride, it shows similar behavior to vacuum, presenting the largest near set S1 BLA and the smallest far set S1 BLA out of the solvents tested.

\subsection{Wiberg Bond Orders}
\begin{minipage}[c]{\textwidth}
  \centering
  \includegraphics[width=5in]{../Paper1/Images/ccl4-5s-widberg.png}
  \captionof{figure}[Wiberg Bond Orders During Adiabatic Dynamics.]{Wiberg Bond Orders for PPV\(_3\)-NO\(_2\) in CCL\(_4\) with 5QM Solvents During Adiabatic Dynamics. }
  \label{fig:bondOrdersAdiabatic}
\end{minipage}\bigskip

The significant effects of the S0-S1 transitions on the Cartesian measurement of BLA encourages the analysis of these bonds' quantum mechanical behavior.
Because the double-bonds elongate and the single-bonds contract, we expect the single-bonds to gain a partial double-bond character and vice versa.
Simple bond ordering does not consider these subtleties of a quantum electronic wave-function.
Fortunately, the quantum mechanical descriptor, the Wiberg bond index, provides a reasonable analogy of the classical Lewis structure a chemist would expect.
Wiberg bond indexes are calculated from the density matrix by 
\begin{equation}
  W_{AB} = \sum_{\mu\in A}\sum_{\nu \in B} | D_{\mu\nu} |^2,
\end{equation}
where \(A\) and \(B\) are indexes of the two atoms, \(\mu\) and \(\nu\) are the atomic orbitals, and \(D\) is the density matrix.
The method sums the electron density shared by both atoms.
If an electron is fully localized on a single atom, the sum of the elements equals zero providing a value that matches our intuition. 

As the bond order increases, we expect the bond to become more rigid and the bond length to shrink.
Figure \ref{fig:bondOrdersAdiabatic} displays the bond order of bonds d1-6 for PPV\(_3\)-NO\(_2\) in carbon tetrachloride with 5 QM solvent molecules included in the QM region.
At time t = 0, the system instantaneously transitions to the first excited state, S1.
The Wiberg bond index then uses the density matrix for S1, leading to a sudden shift in its value.
At S1, the bond orders of d2 and d5 instantaneously drop, and expansion of their bond lengths soon follows.
The larger shifts in the near set correspond to the information found in the BLA analysis.
The interior bonds (d1 and d4) have lower bond indexes than their exterior counterparts (d3 and d6).

\subsection{Torsional Angles}

\noindent
\begin{minipage}[c]{\textwidth}
  \centering
  \includegraphics[width=6in]{../Paper1/Images/dihedrals.png}
  \captionof{figure}[Torsional angles during adiabatic dynamics]{Torsional angle around d1-d3, near, and d4-d6, far, in S1 within various solvents with 20 solvents included in the QM region.}
  \label{fig:dihedralAdiabatic}
\end{minipage}\bigskip

\begin{table}[H]
  \caption{Adiabatic excited state torsional angle Relaxation} \label{tab:adiabaticDihedrals}
  % \begin{center}
  \begin{tabularx}{\textwidth}{XXXXX}\hline
    Molecule    & S0 Near  & S1 Near & S0 Far & S1 Far\\\hline
    Vacuum      & 28.0 \(\pm\) 1.0\(^\circ\) & 12.5 \(\pm\) 0.5\(^\circ\) & 28.5 \(\pm\) 0.9\(^\circ\) & 17.4 \(\pm\) 0.8\(^\circ\)\\
    CCl\(_4\)  & 25.7 \(\pm\) 0.8\(^\circ\)  & 12.5 \(\pm\) 0.6\(^\circ\) & 24.5 \(\pm\) 1.1\(^\circ\) & 14.5 \(\pm\) 0.6\(^\circ\)\\
    CH\(_3\)OH  & 26.4 \(\pm\) 1.0\(^\circ\)  & 11.7 \(\pm\) 0.7\(^\circ\) & 28.1 \(\pm\) 0.9\(^\circ\) & 16.1 \(\pm\) 0.6\(^\circ\)\\
    CHCl\(_3\)  & 25.1 \(\pm\) 0.9\(^\circ\)  & 11.8 \(\pm\) 0.5\(^\circ\) & 26.2 \(\pm\) 1.0\(^\circ\) & 15.7 \(\pm\) 0.6\(^\circ\)\\\hline
  \end{tabularx}
\end{table}

As we've seen, in the ground, A\(_g\) state, the \(\pi\) bonds along the vinylene segments experience significant electronic spin alternation. Immediately after excitation, PPV\(_3\)-NO\(_2\) transitions from this A\(_g\) state to the lowest excited state that possesses B\(_u\) symmetry, which allows the dispersion of the electrons across these segment and a reduction in this alternation. As the single-bonds gain more double-bond-like characteristics, the segment begins to encourage a more planar configuration.  The torsion angle around the vinylene segments is highly coupled to the excited state.\cite{nelson2011nonadiabatic,panda2013electronically}
We can, therefore, following precedent, use the torsion angle around the vinylene segments as the slow nuclear coordinates of PPV\(_3\)-NO\(_2\). \cite{Clark2012,barford2011ultrafast}
Torsion angles around d1 and d3 (near) and d4 and d6 (far) are averaged over 128 trajectories to produce the torsion angles data shown in Figure \ref{fig:dihedralAdiabatic} and Table \ref{tab:adiabaticDihedrals}.

Figure \ref{fig:dihedralAdiabatic} shows these torsional angles for PPV\(_3\)-NO\(_2\) within our test solvents. Once again, at time t = 0, the molecule is excite from S0 to S1.  
Table \ref{tab:adiabaticDihedrals} shows a summary of the torsion angles analysis of all tested solvents after five ps of relaxation after the jump to the first excited state.
The trajectories include 20 solvent molecules within the QM calculations.
A noticeable shift towards a planar geometry occurs in all solvents.
This shift is greatest near the nitro group.
