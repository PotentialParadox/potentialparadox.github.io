\chapter{Spectroscopic Analysis of PPV\(_3\)-NO\(_2\)}

\section{Introduction}
   NEXMD is an efficient program for the simulation of photoinduced dynamics of extended conjugated molecular systems involving manifolds of coupled electronic excited states over timescales extending up to 10s of picosectonds. \cite{sifain2018photoexcited, Bjorgaard2015, case2020a, tretiak02_densit_matrix_analy_simul_elect, malone2020nexmd}
   It includes solvent effects using implicit models. These implicit solvents provide insight into the electrostatic forces, but the use of explicit solvent should provide additional information about the quantum or steric effects , allowing the simulation of electron transfer processes due to the stabilization of charge separation in the excited state.
   Many multichrormophoric molecular systems are soluble in polar solvents such as water, where such simulations could provide sought after insight into effects of charged side groups on the structural sampling, structural rearrangement, and transition density redistribution during electronic relaxation.  

   Adequate sampling of the solvent and solute configuration space, including hundreds to thousands of solvent molecules, currently cannot be achieved by QM calculations alone due to the infeasible computational cost of such extensive systems. \cite{barbatti2011nonadiabatic}
   QM/MM methods exist to allow the computationally expensive QM calculations to be performed only on the area of interest while propagating the rest of the system with cheaper classical dynamics.
   The SANDER program in the AMBER molecular dynamics package performs classical molecular dynamics with the use of periodic boundary conditions with high optimization.
   SANDER with QM/MM performs well with systems with tens of thousands of MM atoms and currently calculates QM/MM with semi-empirical Hamiltonians and DFT; however, previous versions did not provide options for excited-state MD.  

   In this work, we redirect AMBER's SANDER package from its usual semi-empirical QM package, SQM, to a modified NEXMD library.
   SANDER linked to the NEXMD library performs adiabatic dynamics at ground-state and CIS excited-state potential energy surfaces on ~100 atoms at QM and 1000s at MM.
   We apply our method to an analysis of a three-ring para-phenylene vinylene oligomer (PPV3-NO2).
   We look at how explicit solvents affect PPV3-NO2's excited state structure as well as its absorption and emission spectra.

\section{Simulation Methods}

\noindent
\begin{minipage}[c]{\textwidth}
  \centering
  \includegraphics[width=5in]{../Paper1/Images/trajectory_diagram/trajectories.pdf}
  \captionof{figure}[Description of Adiabatic Dynamics]{Description of the Adiabatic Simulation}
  \label{fig:adiabaticDynamics}
\end{minipage}\bigskip

Figure \ref{fig:adiabaticDynamics} visually describes the layout of our simulations performed in vacuum, methanol, choroform, and carbontetrachloride.
First, a molecular dynamics simulation of 320 ps was run in the ground state at 300K (NVT) using the General AMBER Force Field ground state Hamiltonian and periodic boundary conditions, as guided form previous convergence studies. \cite{silva2010benchmark}
128 evenly spaced snapshots were collected from this simulation and used as intial conditions for another 128 individual 10ps-simulations equilibrated at 300K (NVT) using QM/MM ground state Hamiltonian.
QM calculatations were performed using configuration interaction singles (CIS) with the AM1 hamiltonian which has previously been show to provide reasonable accuracy for computational cost. \cite{silva2010benchmark}
QM/MM excited state molecular dynamics simulations were run during 10 ps starting from the final configuration of each of the 128 ground state QM/MM trajectores by vertical excitation to the S1 state. \cite{nelson2012nonadiabatic}
A classical time step of 0.5 fs and a Langevin thermostat with friction constant of 2 ps-1 have been used either for ground state and excited state QMMM simulations. \cite{nelson2012nonadiabatic}
The CIS calculations include the first of excited states of the PPV3-NO2 molecule along with the stated number of solvent molecules closest to the central benzene ring.
We treat all other solvent molecules using classical dynamics.
We restrict the QM solvents from drifting away from the solute and the other MM solvents from drifting closer than these QM solvents.

As shown in Figure, the excited-state density resides towards the center of the molecule on the vinyl groups nearest to the phenyl group.
Charge movements on solvents far from this concentration of density cause negligible energy differences.
To maximize the utility of the QM/MM calculation, we only include the solute and the solvent molecules nearest to this central phenyl group in the QM calculations.
To prevent these solvents from drifting during the trajectories, we implement a simple harmonic restraint using AMBER's NMR restraints.
We select the N solvent molecules based on the proximity of the solvent atom closest to any atom located on the central benzene ring.
We then restrain these solvents using the harmonic constraint on the distance from the center of geometry of the solvent to the center of geometry of the central phenyl group.
We also restrain the solvent molecules not included in the QM calculation from getting closer to the central phenyl than the QM solvents, effectively making a spherical shield around the solute's central phenyl group.
Since the distance between the center of geometries and the closest atoms are not necessarily equal, solvent atoms could initially be pushed either inside or outside this shield during the transition from MM to QM.
However, this push only occurs during the initial equilibration of the QM ground-state calculations excluded from any analysis.
Once the QM calculations begin, these constraints persist throughout all further calculations.
For our CCl4 simulations with the 5 nearest solvents included (CCl4-5QM), the restriction barrier had an average radius of 6.28 \AA with the origin at the center of the central phenyl group.

\section{Results}

\subsection{Bond Length Alternation}
	The structural differences between the excited-state and ground-state of PPV3-molecules are presented clearly by distortions in the C=C and C-C bonds found in the vinylene segment.\cite{tretiak02_densit_matrix_analy_simul_elect, karabunarliev2000rigorous, karabunarliev2000adiabatic, nelson2014nonadiabatic}
	These distortions can be measure by bond length alternation (BLA)

	\begin{equation}
	\frac{d_i + d_e}{2} - d_e,
	\end{equation}

	where \(d_i\) and \(d_e\) are the interior and exterior bonds, and \(d_c\) is the central bond.
	This value represents the differences between the double and single bonds of the vinylene sections.
	The BLA is a descriptor for \(\pi\) bond distributions. \cite{tretiak2002conformational}
	In this system, we analyze the BLAs of two separate bond sets, the bonds d1-3 (near-set) and d4-6 (far-set) seen in Scheme 2. 

	\noindent
	\begin{minipage}[c]{\textwidth} 
	  \centering
	  \includegraphics[width=5in]{../Paper1/Images/bla.png}
	  \captionof{figure}[BLA of bonds during adiabatic dynamics]{BLA of bonds d1-3 (left) and d4-6 (right) during the last 4 ps of S0 dynamics and 10 ps of S1 dynamics.
	    QM energy and force calculations include the 20 solvents nearest the central ring.}
	  \label{fig:bla_adiabatic}
	\end{minipage}\bigskip



\begin{table}[H]
  \caption[Adiabatic Bond Length Alternation]{Bond Length Alternation summary for PPV\(_3\)-NO\(_2\) in various solvents with 20 solvents included in the QM region.}
  % \begin{center}
  \begin{tabularx}{\textwidth}{XXXXXXXXX}\hline
    Molecule   & d\(_1\)  & d\(_2\) & d\(_3\) & BLA\(_{\textbf{near}}\) & d\(_1\)  & d\(_2\) & d\(_3\) & BLA\(_{\textbf{far}}\)\\\hline
    Vacuum     & 1.429     & 1.375    & 1.418    & 0.049              & 1.441     & 1.365    & 1.427   & 0.069\\
    CCl\(_4\)  & 1.427     & 1.376    & 1.417    & 0.046              & 1.441     & 1.365    & 1.426   & 0.068\\
    CH\(_3\)OH & 1.422     & 1.382    & 1.415    & 0.037              & 1.444     & 1.362    & 1.431   & 0.076\\
    CHCl\(_3\) & 1.423     & 1.380    & 1.415    & 0.039              & 1.443     & 1.365    & 1.429   & 0.074\\\hline
  \end{tabularx}
\end{table}


	Figure \ref{fig:bla_adiabatic} show the bond length alternation of PPV3-NO2 in various solvents during the S1 trajectories where the dashed lines represent equilibrated ground state values which are near 0.110 Å for sets d1-3 and d4-6 regardless of solvent enviornment.
	Within the first couple hundred femtoseconds after the excitation to S1, the central bonds expand, while the interior and exterior bonds contract.
	For all solvent environments, this bond restructuring is amplified by close proximity to the amino group, where we see an average drop of 0.07 Å in sets d1-3 compared to a 0.04 Å drop in sets d4-6.
	The strength of this amplification is dependent on the solvent environment where the BLA difference between the far and near sets PP3-NO2 in methanol, 0.034 Å, surpases that found in carbon tetrachloride, 0.022 Å.

	Table 2 presents further details of the S1 BLA simulation.
	In all cases, the exterior bond (d1 and d4) becomes slightly longer than the interior bond (d3 and d6).
	The BLAs from the near and far sets therefore split.
	In the near set, the S0 and S1 BLA converge to 0.1091 Å and 0.00453 Å, respectively.
	In the far set, these numbers are 0.1103 Å and 0.0697 Å, respectively.
	The smaller bond length spread in the near set, along with the lower BLA, suggests more delocalization on those bonds than in the far set. 

	The number of solvents included in the QM calculations had little effect on the convergence of the distances or BLA.
	PPV3-NO2 had similar ground state BLAs of around 0.11 Å for both near and far sets matching results on similar systems regardless of the solvent. \cite{nelson2011nonadiabatic}
	The information presented is averaged over time after relaxation across all trajectories.
	The S1 BLAs varied between the solvents and with the distance away from the NO2 group.
	Among the selected solvents, CH3OH has the smallest near set S1 BLA and largest far set S1 BLA.
	CCl4 has the largest near set S1 BLA, and also the smallest far set S1 BLA and has close to vacuum-like behavior.
	The CH3OH solvent seems to enlarge the BLA changes from the ground to excited states.

\subsection{Wiberg Bond Orders}
\begin{minipage}[c]{\textwidth}
\centering
\includegraphics[width=5in]{../Paper1/Images/ccl4-5s-widberg.png}
\captionof{figure}[Wiberg Bond Orders During Adiabatic Dynamics.]{Wiberg Bond Orders for PPV\(_3\)-NO\(_2\) in CCL\(_4\) with 5QM Solvents During Adiabatic Dynamics. }
\label{fig:bondOrdersAdiabatic}
\end{minipage}\bigskip


    The significant effects of S0-S1 transitions on the Cartesian measurement of BLA encourages the analysis of these bonds' quantum mechanical behavior.
    Because the double bonds elongated and the single-bonds contracted, we expect the single-bonds to gain a partial double-bond character and vice versa.
    Simple bond ordering does not consider these subtleties of a quantum electronic wave-function.
    Fortunately, the quantum mechanical descriptor, Wiberg bond index, provides a reasonable analogy of the classical Lewis structure a chemist would expect.
    Wiberg bond indexes are calculated from the density matrix by 

    \begin{equation}
    W_{AB} = \sum_{\mu\in A}\sum_{\nu \in B} | D_{\mu\nu} |^2
    \end{equation}

    where \(A\) and \(B\) are indexes of the two atoms, \(\mu\) and \(\nu\) are the atomic orbitals, and \(D\) is the density matrix.
    The method sums the electron density shared by both atoms.
    If an electron if fully localized on a single atom, the sum of the elements equals zero providing a value that matches our intuition of a bond. 

    As the bond order increases, we expect the bond to become more rigid and the bond length to shrink.
    Figure \ref{fig:bondOrdersAdiabatic} displays the bond order of bonds d1-6 for PPV3-NO2 in CCl4 with 5 QM solvent molecules.
    At time t=0, the system instantaneously transitions to the first excited state, S1.
    The Wiberg bond index then uses the density matrix for S1, leading to a sudden shift in its value.
    At S1, the bond orders of d2 and d5 instantaneously drop, and expansion of their bond lengths soon follows.
    The larger shifts in the near set correspond to the information found in the BLA analysis.
    The interior bonds, d1 and d4, have lower bond indexes than their exterior counterparts, d3 and d6.

\subsection{Torsional Angles}

\noindent
\begin{minipage}[c]{\textwidth}
  \centering
  \includegraphics[width=4in]{../Paper1/Images/dihedrals.png}
  \captionof{figure}[Tortional Angles during Adiabatic Dynamics]{Tortional angle around d1-d3, near, and d4-d6, far, in S1 within CCl\(_4\)-5QM}
  \label{dihedralAdiabatic}
\end{minipage}\bigskip

\begin{table}[H]
  \caption[]{}
  % \begin{center}
  \begin{tabularx}{\textwidth}{XXXXX}\hline
    Molecule    & S0 Near  & S1 Near & S0 Far & S0 Far\\\hline
    Vacuum      & 28.0 \(\pm\) 1.0\(^\circ\) & 12.5 \(\pm\) 0.5\(^\circ\) & 28.5 \(\pm\) 0.9\(^\circ\) & 17.4 \(\pm\) 0.8\(^\circ\)\\
    CHCl\(_4\)  & 25.3 \(\pm\) 1.2\(^\circ\)  & 12.3 \(\pm\) 0.6\(^\circ\) & 25.1 \(\pm\) 1.3\(^\circ\) & 14.7 \(\pm\) 0.8\(^\circ\)\\
    CH\(_3\)OH  & 26.6 \(\pm\) 0.9\(^\circ\)  & 11.5 \(\pm\) 0.5\(^\circ\) & 28.2 \(\pm\) 0.9\(^\circ\) & 16.2 \(\pm\) 0.8\(^\circ\)\\
    CHCl\(_3\)  & 26.8 \(\pm\) 1.1\(^\circ\)  & 11.8 \(\pm\) 0.5\(^\circ\) & 27.8 \(\pm\) 1.2\(^\circ\) & 16.0 \(\pm\) 0.7\(^\circ\)\\\hline
  \end{tabularx}
\end{table}

    We use the torsion angle around the vinylene segments as the slow nuclear coordinates of PPV3-NO2-molecules following precendent. \cite{Clark2012}
    In PPV3-NO2 systems, the excitation to S1 leads to relaxation towards a nearly planar structure Torsion angle around d1,d3 and d4,d6 are averaged over 128 trajectories to produce the near and far torsion angle data respectively. 

    For our CCl4-5QM example, the dihedral angle around the near set d1 to d2 equilibrates around 23° and 12° in the S0 and S1 states, respectively.
    For the near set, d4 to d6, these values are 23° and 15°.
    Once again, only a noticeable difference in S1.
    The time constants for the S1 dihedral relaxations are around 0.8 ps.
    Relaxation of the dihedral angles occurs by four ps. 

    Table 5 shows a summary of the torsion angles analysis of all tested solvents after five ps of relaxation after the jump to the first excited state.
    The trajectories include 20 solvent molecules within the QM calculations.
    A noticeable shift towards a planar geometry occurs in all solvents.
    This shift is greatest near the nitro group.

\subsection{Potential Energy Relaxation}

\noindent
\begin{minipage}[c]{\textwidth}
  \centering
  \includegraphics[width=5in]{../Paper1/Images/energies.png}
  \captionof{figure}[Potential Energy Relaation During Adiabatic Dynamics]{Potential energy difference from ground state for PPV\(_3\)NO\(_2\) in CCl\(_4\) with 5 QM solvent molecules.}
  \label{fig:energiesAdiabatic}
\end{minipage}\bigskip

 The absorption and the fluorescence properties are judged primarily through the difference between the ground state (S0) and the first excited state (S1) energies.
 The system starts at the S0, where it remains near the bottom of the energy well.
 Figure \ref{fig:energiesAdiabatic} shows the energies for states S0 and S1 averaged over 128 trajectories for PPV3-NO2 in CCL4 with five solvent molecules included in the QM calculations.
 During the first six ps, the system runs on the ground-state S0, and the S0 energies stay near the minimum with small oscillations caused by temperature.
 At the time 0 ps, the system instantaneously hops to the S1 potential energy surface.
 The average energy difference at t=0 between S0 and S1 is 2.93 eV, and it corresponds reasonably well with the peak of the absorption spectrum.
 When the system relaxes on the new surface, the S1 and S0 energies decrease and increase respectively, until the difference between the two is 2.50 eV agreeing with the peak found in the fluorescence spectra.
 Table ? presents the fitted decay of S1 energies using   

\begin{equation}
E = E_d \text{e}^{-t/\tau} + c
\end{equation}
where \(E_d\) is the relaxation energy drop, \(\tau\) the time constant, and c the steady state energy at S1.

\subsection{Spectra}

\noindent
\begin{multiFigure} 
  \addFigure{0.4}{../Paper1/Images/nquant_abs_comparison.png}
  \addFigure{0.4}{../Paper1/Images/nquant_flu_comparison.png}
  \addFigure{0.4}{../Paper1/Images/spectra_abs_compared.png}
  \addFigure{0.4}{../Paper1/Images/spectra_flu_compared.png}
  \captionof{figure}[Fluorescence and Absorption Spectra]{PPV\(_3\)-NO\(_2\) absorption and fluorescence spectra A-B) in CCL\(_4\) with varying number of QM solvents. C-D in various solvents with 20 included in QM region.}
  \label{fig:}
\end{multiFigure}\bigskip


    Energies, coordinates, and dipole information are acquired every ten steps.
    Equilibration times from either MM to QM state or from the S0 to S1 state range from 2-4 ps.
    We exclude the first four ps of each trajectory in the calculation of the spectra in data analysis for the absorption and emission analysis. 

    Previous studies have analyzed solvatochromic shifts in conjugated substituted PPV3-NO2 molecules with the NEXMD program in implicit solvents.
    Results with NEXMD by TD-AM1 were redshifted from the experimental results, while single-point calculations using TD-CAM-B3LYP provided by G09 in the same implicit solvent were blue shifted.
    Other NEXMD computations have shown comparable redshifts in spectra of similar molecules in implicit solvents compared to experiment. \cite{Bjorgaard2015}
    We performed similar calculations in this paper; however, in explicit solvent.
    We compare the results to those presented in implicit solvent.

    We collect the vertical excitation dipoles and oscillator strengths between the ground state S0 and S1 every five fs during the steady-state of each trajectory to produce the absorption/emission spectra of PPV3-NO2.
    We sum over excitation states averaged over the geometries and broaden the spectra using a Gaussian bin function with FWHM=0.16 eV corresponding to a 100 fs FWHM laser excitation.
    We normalized it such that the maximum absorption is 1. 

    Figure ? presents the absorption spectra for PPV3-NO2 in select solvents and vacuum.
    The shown absorption has contributions from the nine lowest energy excited states, though the S1 state is the primary contributor to the spectra.
    We found the number of solvent molecules included in the QM region caused only minor deviations in the spectra, with the largest variance (0.02~eV) occurring between the 20QM and MM CCL4 systems, as such, in figure 12, we only present results from trajectories with 20 QM solvents.
    All solvent results are redshifted from those in vacuum matching findings in previous works.
    The absorbance within methanol and chloroform were very similar, with a peak shift from vacuum of -0.04 eV.
    Within carbon tetrachloride, this shift increases slightly more to-0.06 eV. 

    Aligning with previously reported results, the fluorescence calculations found in figure 13 show an overall more intense redshift from vacuum, along with a more significant dispersion among the solvents. \cite{Park2013}
    The smallest shift, at -0.06 eV, occurs in carbon tetrachloride, while the largest, at -0.12 eV, occurs in methanol.
    Previous works have demonstrated that the energy levels of PPV3-NO2 are further stabilized by more polar solvents, a feature clearly seen by our results.

The Non-adiabatic Excited State Molecular Dynamics package (NEXMD) that SANDER is linked to for the excited state calculations done in this paper is designed to perform molecular dynamics simulations outside the Born-Oppenheimer approximations using the FSSH in implicit solvents.
The program has been tested well and is helpful in gaining insights into the dynamics of molecules prone to photo-excitations.
However, simulations in explicit solvents have been shown to produce qualitatively different results than in implicit solvent mediums.
By linking AMBER's SANDER to NEXMD, we produced a quick and efficient tool to simulate explicit solvent behavior on adiabatic excited state dynamics.
