During photo-excitation and relaxation processes, we can no longer rely on the Born-Oppenheimer approximation.
  Non-adiabatic dynamics methods are therefore essential in the simulation of this behavior.
  The Non-adiabatic-Excited-state Molecular Dynamics Package (NEXMD) is designed and optimized to perform these calculations in a highly efficient manner. However, its use is restricted to implicit solvent.
  SANDER, a molecular dynamics program found in the AMBER molecular dynamics package, can perform hybrid quantum mechanical, classical dynamic (QM/MM) dynamics.
    In this work, we build a library of useful routines from NEXMD and present a method to link this library to SANDER to provide excited-state adiabatic and non-adiabatic QM/MM simulations.
  We test this new methodology on a derivative of an organic conjugated polymer Poly(p-phenylene vinylene) (PPV3-NO2).
  We first investigate the steady-state characteristics of PPV3-NO2 at the ground state and lowest excited state in varying solvents. This analysis only required excited state adiabatic calculations, and we compare the results to those from experiment and implicit solvent experiments.
  We then apply the non-adiabatic routines and inspect the behavior of the excited state population decays.
For the non-adiabatic dynamics simulations, we restrict our calculations to the use of methanol as our solvent and find that this solvent's inclusion leads to similar behavior to that found in implicit solvents with a similar dielectric constant.
