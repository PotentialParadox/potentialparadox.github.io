\documentclass[editMode]{ufdissertation}\sloppy
  \usepackage{tikz}%       tikz is used by almost everyone, but certainly by me for this.
  \usepackage{bm}%         bm is needed in order to boldface mathematical symbols
  %% Uncomment the relevant line below if you have tables or figures.
  \haveTablestrue%        Uncomment this if you have tables in your thesis.
  \haveFigurestrue%       Uncomment this if you have figures in your thesis.
  % \haveObjectstrue%       Uncomment this if you have Objects in your thesis. This is almost certainly not the case however.

  \title{The Implementation of QM/MM on Non-Adiabatic Dynamics Using AMBER and NEXMD}%  Put your title here.

  \degreeType{Doctorate of Philosophy}%   Official name of your degree; eg "Doctorate of Philosophy".
  \major{Mathematics}%                    Your official Department
  \author{Dustin Tracy}%                  Your Name
  \thesisType{Dissertation}%              Dissertation (PhD) or Thesis (Masters)
  \degreeYear{2021}%                      Intended graduation year (not the year you submit the thesis)
  \degreeMonth{May}%                   Month of graduation should be May, August, or December.
  \chair{Adrian Roitberg}%                   Chair and Cochair (see comment block above).


  %%%%%%%%%%%%%%%%%%%%%%%%%%%%%%%%%%%%%%%%%%%%%%%%%%%%%%%%%%%%%%%%%%%%%%%%%%%%%%%% 
%%% For each of the following, type in the name of the file that contains each section. 
%       They are assumed to be tex files, but if they aren't the command takes an optional argument for the extension.
%       So, you could load dedication.tex as your dedication file using \setDedicationFile{dedication}
%       You could load dedication.txt instead with \setDedicationFile[txt]{dedication}.
%       NOTE: For some compilers they may or may not add a .tex to the end of the file automatically.
%           If you get a "couldn't find dedication.tex.tex" type error, try the command with an empty optional argument,
%           e.g. \setDedicationFile[]{dedication}
%%%
%%%%%%%%%%%%%%%%%%%%%%%%%%%%%%%%%%%%%%%%%%%%%%%%%%%%%%%%%%%%%%%%%%%%%%%%%%%%%%%%

%%% These are REQUIRED sections; easiest to do via these commands.

\setDedicationFile{dedicationFile}%                 Dedication Page
\setAcknowledgementsFile{acknowledgementsFile}%     Acknowledgements Page
\setAbstractFile{abstractFile}%                     Abstract Page (This should only include the abstract itself)
\setReferenceFile{referenceFile}{amsplain}%         References. First argument is your bibtex source file
%                                                       the second argument is your bibtex style file.
\setBiographicalFile{biographyFile}%                Biography file of the Author (you).

%%% These are NOT required, so only use them if you actually need/have them.

% \setAbbreviationsFile{abbreviations}%           Abbreviations Page
% \setAppendixFile{appendix}%                     Appendix Content; hyperlinking might be weird.
% \multipleAppendixtrue%                          Uncomment this if you have more than one appendix, 
%                                                   comment it if you have only one appendix.


%%%%%%%                     End of File Assignment
%%%%%%%%%%%%%%%%%%%%%%%%%%%%%%%%%%%%%%%%%%%%%%%%%%%%%%%%%%%%%%%%%%%%%%%%%%%%%%%%

\begin{document}
%%%% Here you just need to include/input your actual work. 
%       The above files (dedication, acknowledgement, titlepage, etc etc) will all be added for you 
%       using the files you assigned above. 
%       If you want to input the above files manually you can comment out the \setFILE command above 
%       and use \input or \include here. Generally you want to use \include to get your pagebreak.
%       NOTE: If you input manually you will have to do some/all the formatting manually.



\chapter{Introduction} \label{introduction}

\section{Prologue}

The effects of light on the physical properties of materials has maintined the interest of mankind for as long as history itself.
Humans most likely noticed power of the sun to turn their skin red and itchy far before they even developed language.
Records show the interest of reducing the bleaching dyes.
The documents describing the mirror of Archimedes demonstrates that humans desire to harness this power dates back at leas multiple millinea.
Our understanding began to formalized in the late 1700s when Priestly experiments shined light on the processes of oxidations and sparked a curiosity that led to further experimentations with photosynthesis and photochemistry in general.
Since then, researchers have further advanced our knowledge of these effects and our ability to harness the power of light.

The ability to model these photo-energetic non-adiabatic dynamics has recently become more feasable.
We have used this ability to continue our long pursuit to understand organic photosythesis. \cite{zheng2017photoinduced,caycedo2010light}
The search for how to efficiently create and utilize sythetic organic phtosythesizer has also been a focus of interest. \cite{balzani2008photochemical,engel2007evidence}
Studies with non-adiabatic dynamics have been used to study possible light harvesting technologies. \cite{ishida11_effic_excit_energ_trans_react,katan2005effects}

A similar process can also illuminate our understanding and production of efficient custom light emitting diodes. (cite: park 1-2)
This type of modeling can also help with understanding photo-detection.
Recent works have helped understand how the the protein rhodopsin behaves in the human eye.\cite{weingart2012modelling}
Continued research can help develop more sensitive or enery efficient optical sensors. 
The modeling of theses types of dynamics currently boasts a broad academic and industrial interest. \cite{tavernelli2010nonadiabatic,tavernelli2015nonadiabatic,nelson2020non}
Ultra fast proton transfer on the time order of femtoseconds have sparked much interest in last few decades.\cite{schwartz1992direct}

\section{Qualitative Overview of Non-Adiabatic Dynamics}

\subsection{Energy Transfer}

\noindent
	  \begin{multiFigure} 
	    \addFigure{0.45}{../Oral/Images/photoexcitation.png}
	    \addFigure{0.45}{../Oral/Images/pes_chart_zoomed.png}
	    \captionof{figure}{Diagrams describing the behavior of a molecule throughout an photo-excitation event.}
	    \label{fig:jablonski}
	  \end{multiFigure}
\bigskip

Figure \ref{fig:jablonski} A shows whats referred to as a Jablonski diagrams.
S\(_0-2\) represent the potential energy surfaces for the three lowest singlet states.
T\(_1\) represents the first excited triplet state.
No vibrational or rotational modes are shown since we will treat these classically.
Immediately after an electron photon absorption, the molecule is promoted to an excited state, as can be seen by the purple arrow.
This excited state could either the one immediately above it, or it could be one the many above that one.
The decision of which state to excited to is determined by the energy of the excitation and oscillator strength.

Once the molecule is at this excited states, ignoring high temperature, it will relax back towards the ground state.
There are two primary mechanisms through which this can occur.
The first is by releasing the energy thermally either throughout the rest of the molecule or to the environment. This method is referred to as internal conversion and manifests as reductions to the vibrational and rotational modes.
The second is through photon-emission.
A photo-emission process from the first excited state to the ground state is referred to as fluorescence, and can be seen by the green arrow in the figure.
Fluorescence occur over period of nanoseconds.
Tranistion processeses from singlet states to the triplet states are possible dependent on the strength of the spin-orbit coupling, in a process called intersystem conversion.
Photo-emmission from the triplet state to the ground state would be called phosphorescence.
Phosphorescence is relatively very rare compared to fluorescence with time order ~1s.
For this reason we do not consider this behavior in our simulations.

Kashas rule states that photon-emission occurs only in appreciable yields from the lowest excited state to the ground state.\cite{Kasha1950}
This rule suggests that in most cases where an electron is excited to a state beyond the first excited state, that electron will have to relax to the first excited state by means of internal conversion.\cite{shenai2016internal}

Figure \ref{fig:jablonski} B, is a zoomed in picture of the portion of Figure 1b surrounded by the orange circle.
When the molecule is excited to S2 through photo-excitation, it will begin to relax along S2's potential energy surface following the orange arrow.
In reality, this process would be quantized and occur as a gradual reduction in the vibration and rotational modes.
In our simulations though, we treat these reduction classically and the molecule can move smoothly along the potential energy surface of each excited state. 
However, eventually the molecule traversing the potential energy surface of S2 will cross the potential energy surface of S1.
At these crossings, there is generally strong couplings between the two states.
This coupling allows the molecule to transition from S2 to S1.
A choice now needs to made whether to stay on potential energy surface of the S2 or switch to S1.

In computation chemistry it is common to assume that electrons move significantly faster than nuclei and treat the nuclei as parameters to the equations used to solve for electronic behaviors.
This assumption is known as the Born-Oppenheimer approximation and forces the molecule to traverse along a single potential enery surface making it impossible for trasitions from one excited state to another to occur.
Simulations of traversals restricted to a single potential energy surface is referred to as adiabatic dynamics.
Simulations that allow such crossings are non-adiabatic.

During ultra-fast photovolatic processes, the Born-Openheimer appoximation breaks, and accounting for non-adiabatic behavior become necessary.
These situations occur frequently within processes of interest to photochemistry and photophysics.
For example, the excitation to a non-equlibrium state followed by a relaxation through internal conversion is a process common to processes such as photosynthesis, solar-cell photo-absoprtion, optical detectors, and the excitation of the visual nerve.
Photon absorption is also a requirement in certain reactions that need that last little kick.\cite{vincent2016little}
The S\(_1\) and S\(_2\) lines figure \ref{fig:jablonski} represent crossing between potential energy surfaces.

\noindent
       \begin{multiFigure} 
	 \addFigure{0.45}{Images/probabilities.png}
	 \addFigure{0.45}{Images/ehrenfestVsTully.png}
	 \captionof{figure}[Surface Hopping vs Mean-Field]{A visual description describing the difference between surface hopping and mean-field. A) The probabilities states S1 and S2. B) The potential energies of trajectories over time. Dashed lines represent represent the potential engergies of S1, S2, and the probability weighted average during the Ehrenfest trajectory. Solid lines represent two sepearte surface hopping trajectories.}
	 \label{fig:surfaceHoppingVsMeanField}
       \end{multiFigure}
\bigskip

The two most common methods to extend the Born Oppenheimer appoximations are through a mean field, ofter referred to as Ehrenfest, or through molecular dynamics with quantum tranistions (MDQT).\cite{Hammes-Schiffer1994}
Alternative methods using mixed quantum-classical dynamics do exist and are used in the field. \cite{habershon2013ring,kapral2006progress}
In Ehrenfest methods, the forces acting on the molecule at any timestep is the population weighted average of the forces acting at all relevant excited states.
In MDQT methods only the forces of one state is used for any single time-step. \cite{prezhdo1997evaluation}
Between timesteps, the molecule is allowed to transition between states.
To simulate state populations, MDQT methods employ a swarm of independent trajectories.
Each trajectory is given a different random seed and allowed to hop between states based on the non-adiabatic couplings.
Study of the system's behavior is then done based on the statistics of the ensemble.

Figures \ref{fig:surfaceHoppingVsMeanField} A and B attempt to show the practical differences between these two methods.
The population chart on the left shows the probability of being in states S1 and S2 at some arbitrary time.
These probabilities merge to around 0.5 each at around the halfway point.

The right figure presents arbitrary state energies over the same time frame for this trajectory.
The dashed lines represent the energies along the Ehrenfest trajectory.
Blue and red represent the S2 and S1 energies repectively.
The black dashed line represents the Ehrenfest mean-field energy determined as the population weighted average energies of S1 and S1.
As the probability of state S2 drops from one, the mean-field energy diverges from that of S2.
Eventually the mean field energy becomes the average of a S1 and S2.

The solid lines represent the the energies along two separate surface hopping trajectories.
At around the halfway point, the trajectory SH-S1 hops from the S2 to S1.
Trajectory SH-S2 remains on S2.
Because these trajectories are allowed to be moved by forces generated at their respective potential energy surfaces, their energies will in general be lower than their mean field counterparts.
Notice that the average energy of the hop trajectories will also diverge from the Ehrenfest method.

\noindent
\begin{minipage}[c]{\textwidth}
  \centering
  \includegraphics[width=\textwidth]{./Images/naCrossings.png}
  \captionof{figure}[Regions of Non-Adiabatic Couplings]{Periods of trajectories where there is in general weak and strong state couplings between states S1 and S2 and well as region where the energies of S1 and S2 cross.}
  \label{fig:naCrossings}
\end{minipage}\bigskip

In this work, we model the interstate transitions using the MDQT algorithm, Tully's Fewest Switches.
The probability of hopping from one state to another is proportional to the coupling between the states known as the nonadiabatic coupling.
These nonadiabatic coupling are dependent in part on the energy differences between the states, and the nuclear velocities.
Figure \ref{fig:naCrossings} shows three approaches from potential surfaces S1 and S2.
Assume that the molecule is originally on state S2.
When the the energy differences are relatively large, with a shallow approach as in the left figure, the coupling is weak, and hops become unlikely.
When the aproach is steep, and the energy difference small, the nuclear velocities no longer become negligible, the Born Oppenheimer approximation breaks, strong coupling exists, and a respective hop become likely.
In the far right figure, the energies of the two states cross.
In general states in molecular dynamics programs are refered to based on their energy orderings.
In this situation, the orderings of these potential energy sufaces swicth and S1\(\rightarrow\)S2 and vice versa.
If no adiabatic hopping occurs, the molecule remains on the same potential energy suface.
However, the energy levels will have have switched and we need to ensure that molecule traverses along the new S1 state.
This can be done by comparing overlaps between the states between timesteps.

\subsection{Solvent Effects}
The determination of which state to excited to is strongly affected by the transition dipole moments.
These transition dipole moments are sensitive to polarization from external electronic fields or charges
The energy differences between the excited states can also be affected by these external charges due to (de)-stabalization of the dipoles.
Systems with strong electric fields occur frequently in biological systems. (cite: furukawahino 41-44)
These electric field can have profound effect on the steady state fluorescnce and absorption spectra. \cite{park2013}
The solvents in these systems can extend or shield these effects.
Solvents provide a large source of external charges that can significantly affect the non-adiabatic behavior and characteristics of a molecule.\cite{furukawa2015external}

Multiple methods to have been proposed and used to simulate these non-adiabatic processes.
These methods include treating the nuclear coordinates quantum mechanically or simiclassically, or by using a hybrid quantum mechanical, classical treatment to account for the non-adiabaticity.
One of the more popular version of the latter, and the one which we use in this work, is Molecular Dynamics with Quantum Transitions (MDQT), were the system propogates classically along adiabatic potential energy surfaces, but a quantum evalutation is made at each time step to determine whether to transition to another state.

For many areas in which nonadiabatic dynamics simulations would be of interest, solvents play a crucial role.
\cite{bagchi1989dynamics,woo2005solvent}
    In situations where ultrafast electronic relaxations occur, the electronic decay is often faster than the time for the solvent to equilibrate.
    As such, Implicit solvents, which adjusts instantaneously to any changes, become imprecise approximation.
    However performing non-adiabatic dynamics on such large systems is far too computationally expensive.
    To ease the computational cost we can employ QM/MM methodologies to perform the non-adiabatic calculation only on the areas of interest.
    Similar methods have been employed in the study of retinal photochemistry and organic semiconductors.\cite{weingart2012modelling,demoulin2017fine,heck2015multi,bayliss1954solvent}%\cite{demoulin2017fine, weingart2012modelling, heck2015multi}
    In this work we implement a new method of performing non-adiabatic QM/MM using the SANDER package AMBERTOOLS combined with the high performance Non-Adiabatic simulator NEXMD.
    We further analyze the effects of including near solvent molecules within the QM region.

\subsection{QM/MM}
	\begin{multiFigure} 
	\addFigure{0.4}{../Oral/Images/qm_mm.png}
	\addFigure{0.4}{../Oral/Images/qm_mm_pme.png}
	\captionof{figure}[QM/MM Diagram]{a) single cell. b) representation of the periodic nature of the system.}
	\label{fig:QMMMDiagram}
	\end{multiFigure}
\bigskip

	In the previous sections we have discussed how quantum mechanics can be used for chemical calculations.
  However, in many applications, the accuracy of QM is not needed and more computationally cheaper method would be more appropriate.
	For these situations many computational chemist use classical electrical force field dynamics, treating atoms as point charges.
	QM/MM was developed to manage computational costs by separating a calculation into a quantum mechanical (QM) region and a classical mechanical (MM) region.\cite{warshel1976theoretical,Karplus2014}
	This allows the user to have the accuracy where needed while not wasting resources on unwanted calculations such as the dynamics of water molecules far from the protein of interest.
	For the vast majority of our calculations, we will have a QM solute and a few nearby QM solvents surrounded by MM solvents.

	Figure \ref{fig:QMMMDiagram} gives an example of a QM/MM systems.
	The atoms of the drawn out molecule will be described at the QM level of theory.
	The MM atoms in the volume immediately surrounding the molecular, label QMCut, will be the MM atoms included in equation \ref{eq:qmmm}.
	To simulate a solute in solvent, we treat the provided box as a cell, that is repeated infinitely many times.
	Particle Mesh Ewald calculations are then used to calculate the long distance interactions of the periodic boxes.
	This is performed by treating the charge and potential in the long range, inter box distances, as sums in Fourier space.\cite{Darden1993}
	Note that the QM region must be treated as single point charges for this calculations.
	The Mulliken charges of the current state are used for these calculations.
	Once the sums are complete, a fast Fourier transform is performed to obtain energy and forces.
	Charges from the MM region outside QMCut, will be used to provide a Particle Mesh Ewald correction to the new Fock Matrix.\cite{Walker2008}

	Long range interaction, from those outside the cutoff, considered vital for the understanding of solvent effects, are treated using SQM’s implementation of Particle Mesh Ewald.
	Trajectories use periodic boundary conditions to simulate an explicit solution, treating the system box as cells repeated infinitely many times in all directions.
	Particle Mesh Ewald calculations then determine the long-distance interactions of these periodic boxes, treating the charges and potentials in the long-range inter-box distances as sums in Fourier space treating atoms in the QM region of these calculations as Mulliken point charges.
	Once the sums are complete, SQM performs a fast Fourier transformation to obtain the long-range corrections to the energy and forces.

\section{Organic Conjugated Molecules}
Conjugate organic polymers have been shown to exhibit ultra-fast exciton decay.\cite{nelson2018coherent,Fernandez-Alberti2009}
Studies have been performed to determine whether we can sythesize unidirectional energy transfers in these systems.\cite{soler2012analysis,soler2014signature,Galindo2015,FernandezAlberti2010,FernandezAlberti2012}
The have a dense manifold of electronic states.
They have strong electron-phonon couplings. (cite: sifain2018photoexcited 14-16)
They have photophysical properties that are rare (cite: sifain2018photoexcited 25,26)
Due also in part to their low cost of production a heavy interest has been show in using them for technological development. (cite: sifain2018photoexcited 17-24)
Experimentally, these molecules are studied either in solution or as solid state samples.
These types of scenarios have been too computationall expensive to siulate explicitely, and have only recently been studied using implice solvents. (cite: sifain and josiahs)

\section{Overview}
In chapter 2 we go into the theoretical methods employed to simulated the previously discossed processes.
In chapter 3 describe discuss the computation details in our implementationdescribe discuss the computation details in our implementation.
In chapter 4 we apply our methodology to investigate the steady state absorption and fluorescence experienced by PPV\(_3\)NO\(_2\) in various solvents.
We also investigate the change in behavior caused by including solvents in the QM region.
In chapter 5 we apply the non-adiabatic methodoly to analyze the effects included QM/MM solvents have the non-adiabatic relaxation of PPV\(_3\)NO\(_2\) 
Finally, in chapter 6 we summarize our findings and suggest possible routes for future work.
% Modified from old template.
\chapter{Theoretical Methods} \label{theoreticalMethods}

\section{Secular Determinants}\label{secular}

The goal of computational chemistry is to solve the Schrodinger equation.
Solving it completely is only possible for very small subsets of possible situations.
In most cases, significant approximations must be made.
One of the more common such approximations, is the appoximation of the atomic basis functions.
If we write the Schodinger equation as

\begin{equation} \label{eq:oneeenergy}
E(\Psi) = \frac{\left<\right.\Psi\left|\right.\bm{H}\left.\right|\Psi\left>\right.}{\left<\right.\Psi\left.\right|\left.\Psi\left.\right.\right>}
\end{equation}

where $\mathbf{H}$ the Hamiltonian, $E$ is the energy of the
system, and $\Psi$ is a wavefunction that describes the system.
$\Psi$ will be some linear combination of functions $\Psi=\sum_{i}c_i\psi_i$.
We can now expand the numerator and denominator of the right-hand side of equation \ref{eq:oneeenergy}.

\begin{align}
  \label{eq:variation1}
  \left<\right.\Psi\left|\right.\bm{H}\left.\right|\Psi\left>\right.&=
								      \left( \sum_{i} c_i \psi_i \right) \mathbf{H} \left( \sum_j c_j \psi_j \right) &
																		       \left<\right.\Psi\left.\right|\left.\Psi\left.\right.\right>&=
																										     \left( \sum_{i} c_i \psi_i \right) \left( \sum_j c_j \psi_j \right)  \\
								    &= \sum_{ij} c_{i}c_j H_{ij} & &= \sum_{ij} c_{i}c_j S_{ij} 
  \label{eq:variation2}
\end{align}

Taking the partial derivatives of both sides with respect to $a_i$ in
equation \ref{eq:variation2} provides us with

\begin{align}
  \label{eq:variationexpansion}
  \frac{\partial}{\partial c_{\alpha}}
  \left<\right.\Psi\left|\right.\bm{H}\left.\right|\Psi\left>\right.&=
								      2c_\alpha H_{\alpha \alpha} + \sum_{\alpha j \neq \alpha} 2c_j H_{\alpha j} &
																		    \frac{\partial}{\partial c_{\alpha}}
																		    \left<\right.\Psi\left.\right|\left.\Psi\left.\right.\right>&=
																										  2 c_\alpha S_{\alpha\alpha} + \sum_{\alpha j \neq \alpha} c_j S_{\alpha j}
\end{align}

If we multiply both sides of equation \ref{eq:oneeenergy} by
$\left<\right.\Psi\left.\right|\left.\Psi\left.\right.\right>$ and
take the partial derivative with respect to $c_{\alpha}$,

\begin{align}
  \frac{\partial}{\partial c_{\alpha}}
  \left( E \left<\right.\Psi\left.\right|\left.\Psi\left.\right.\right> \right)&=
										 \frac{\partial}{\partial c_{\alpha}}
										 \left<\right.\Psi\left|\right.\bm{H}\left.\right|\Psi\left>\right. \\
  \label{eq:variation3}
  E \frac{\partial \left<\right.\Psi\left.\right|\left.\Psi\left.\right.\right>}{\partial c_{\alpha}}
  + \left<\right.\Psi\left.\right|\left.\Psi\left.\right.\right> \frac{\partial E}{\partial c_{\alpha}} &=
													  \frac{\partial}{\partial c_{\alpha}}
													  \left<\right.\Psi\left|\right.\bm{H}\left.\right|\Psi\left>\right.
\end{align}

Now we minimize $E$ by rearranging equation \ref{eq:variation3}

\begin{equation}
  \frac{\partial E}{\partial c_{\alpha}} =
  \frac{1}{\left<\right.\Psi\left.\right|\left.\Psi\left.\right.\right>}
  \left[
    \frac{\left<\right.\Psi\left|\right.\bm{H}\left.\right|\Psi\left>\right.}
    {\partial c_{\alpha}}
    -E \frac{\left<\right.\Psi\left.\right|\left.\Psi\left.\right.\right>}
    {\partial c_{\alpha}}
  \right] = 0
\end{equation}

Substituting our results from equation \ref{eq:variationexpansion} and
dividing by common multipliers, we find

\begin{equation}
  c_{\alpha} H_{\alpha \alpha} + \sum_{\alpha j \neq \alpha} c_j H_{\alpha j} -
  E \left( c_{\alpha} S_{\alpha \alpha} + \sum_{\alpha j \neq \alpha} c_j S_{\alpha j} \right) = 0
\end{equation}

\begin{equation}
  c_{\alpha} H_{\alpha \alpha} + \sum_{\alpha j \neq \alpha} c_j H_{\alpha j} -
  E \left( c_{\alpha} S_{\alpha \alpha} + \sum_{\alpha j \neq \alpha} c_j S_{\alpha j} \right) = 0
\end{equation}

which is often referred to as the matrix form of the Schrodinger
equation.  A more intuitive understanding of the equation may be had
if we expand out for $\alpha=1-3$.

\begin{equation} \label{eq:SchrodingerMatrix}
  \begin{bmatrix}
    H_{11}-ES_{11} & H_{12}-ES_{12} & H_{13}-ES_{13} \\
    H_{21}-ES_{21} & H_{22}-ES_{22} & H_{23}-ES_{23} \\
    H_{31}-ES_{31} & H_{32}-ES_{32} & H_{33}-ES_{33}
  \end{bmatrix}
  \begin{bmatrix}
    c_1 \\
    c_2 \\
    c_3
  \end{bmatrix} = 0
\end{equation}

This equation can be rewritten simply as $Hc=ESc$. The determinant
$\left| H-ES \right|$ is known as the secular determinant, with
eigenvalues corresponding to the energies of the molecular orbitals,
whose characteristics are determined by the coefficients in the
corresponding eigenvector.\cite{engel2012quantum}

\section{Hartree Fock}
    Before we can solve the secular equation we need to know our
    Hamiltonian.  We begin with the generalized Hamiltonian of a
    molecular system,\cite{engel2012quantum}

    \begin{align} \label{eq:fullhamiltonian}
      \begin{split}
      \bm{H} =& -\frac{\hbar^2}{2m_e}\sum_i^{electrons}\nabla_i^2-\frac{\hbar^2}{2}\sum_{A}^{nuclei}\frac{1}{M_{A}}\nabla_{A}^2 - \frac{e^2}{4\pi\varepsilon_0} \sum_i^{electrons}\sum_A^{nuclei}\frac{Z_A}{r_{iA}} \\
      & + \frac{e^2}{4\pi\varepsilon_0}\sum_{i}^{electrons}\sum_{j<i}^{electrons}\frac{1}{r_{ij}} + \frac{e^2}{4\pi\varepsilon_0}\sum_{A}^{nuclei}\sum_{B<A}^{nuclei}\frac{Z_AZ_B}{R_{AB}}
      \end{split}
    \end{align}

    where $n$ is summed over all the nuclei, and the $i$ and $j$ are summed over the electrons. 
    With this Hamiltonian, the secular equation is near impossible to solve without some approximations.
    The one most relevant to our work is the adiabatic approximation, also known as the Born-Oppenheimer approximation, where because the electrons move so much quicker than the nuclei, we can set the second term of equation \ref{eq:fullhamiltonian} to zero and the last term to a constant. We can then rewrite the electron as behaving parametrically on the coordinates of the nuclei such that our wavefunction $\Psi_{total} = \sum_\alpha\psi_\alpha^{electron}(r;\mathbf{R})\psi_\alpha^{nuclei}(\mathbf{R})$.
    The potential energy surface then, can be extrapolated by applying the electronic Hamiltonian $H_e$ to the wavefunction and then adding nuclear repulsion, for an array of nuclear geometries.
    In the mean field approximation, each electron feels the average potential of all the other electrons, such that the fourth term in our total Hamiltonian becomes $\sum_i^{electrons} V_{average}(i)$
    The electronic parts the Hamiltonian are now decoupled, and the total Hamiltonian can now be written as a sum of individual electron Hamiltonian's plus a nuclear-nuclear repulsion constant.
    In actuality the electrons of one orbit will effect electrons of the orbit of another.
    The electrons will want to avoid each-other and their paths will change accordingly thereby reducing the overall energy.
    This approximation to the method fails to take this into account.
    We call the difference between the actual energy $E$ and the Hartree-Fock energy $\epsilon$ the
    coulomb correlation energy $E_{corrrelation}$.
    %There have been numerous ways developed to help alleviate this problem, including perturbation theory, coupled cluster theory, and higher lever configuration interaction.

    The total many electron wavefunction must satisfy the Pauli-Exclusion principle.
    We can fulfill that requirement, if we assume that it is a single slater-determinant of molecular orbitals.

    \begin{equation} \label{eq:slater-determinant} \phi(\bm{r};\bm{R}) =
      \left|p \cdots s\right> = \frac{1}{\sqrt{N}}
      \begin{vmatrix}
	\chi_{p}(\bm{r}_1) & \cdots & \cdots \chi_{s}(\bm{r}_1) \\
	\vdots             & \ddots         &       \vdots      \\
	\chi_{p}(\bm{r}_n) & \cdots & \cdots \chi_{s}(\bm{r}_n)
      \end{vmatrix}
    \end{equation}

    where $\phi$ is the electron coordinates that depend parametrically on the
    nuclear coordinates.  The $p \cdots s$ are the subscripts of the
    molecular orbitals, and $1 \cdots n$ are the indices for the
    electrons.

    Finally, things simplify greatly if the molecular orbitals are
    othornormal to each other. $\left<\right.i\left|\right.j\left>\right. = \delta_{ij}$.
    Intuition tells us that because the Hamiltonian is an operator that
    acts on at most 2 electrons at a time, and the electron orbitals
    are orthonormal, any perturbation beyond 2 will integrate to 0.  In
    fact, there's a whole set of rules to reduce electron integral
    summations called the Slater-Condon rules.\textbf{CITE}

    \begin{enumerate}
    \item
      $ \left | \cdots mn \cdots \right > \rightarrow \left | \cdots mn
	\cdots \right > \Rightarrow \sum_i \left< i \right| h \left| i
      \right> + \frac{1}{2} \sum_{ij} \left< ij \right|\left| ij \right> $
    \item
      $ \left | \cdots mn \cdots \right > \rightarrow \left | \cdots pn
	\cdots \right > \Rightarrow \left< i \right| h \left| i \right> +
      \sum_{i} \left<mi \right|\left| pi \right> $
    \item
      $ \left | \cdots mn \cdots \right > \rightarrow \left | \cdots pq
	\cdots \right > \Rightarrow \left < mn \right | \left | pq \right
      > $
    \item
      $ \left | \cdots lmn \cdots \right > \rightarrow \left | \cdots pqr
	\cdots \right > \Rightarrow 0 $
    \end{enumerate}

    Using these rules and a bit of algebra the Hamiltonian simplifies to
    what's called the Fock operator with elements
    \begin{equation}
      F_{\mu\nu} = \left< \mu \right| -\frac{1}{2}\nabla^2 \left| \mu \right>
      - \sum_{k}^{nuclei}Z_{k} \left< \mu \right| \nu \left> \right.
      + \sum_{\lambda \sigma} P_{\lambda \sigma}
      \left(
	\left< \mu \lambda \right| \nu \sigma \left>\right.
	- \frac{1}{2} \left< \mu \nu \right| \lambda \sigma \left>\right.
      \right)
    \end{equation}

    which can be substituted for $H$ in equation \ref{eq:SchrodingerMatrix} to produce the Roothan-Hall equation $\mathbf{Fc}=\varepsilon\mathbf{Sc}$, where $\varepsilon$ has replace $E$ to be the orbital hartree-fock energies.\textbf{CITE}
    We simplify this further by using the semi-empirical AM1, which uses predetermined factors for the four term integrations.\textbf{CITE}
    We can now apply the variational method to determine the coefficient of the wavefunction.
    First, a trial density function is chosen, which is equivalent to a trial coefficient vector.
    We then solve the Roothan-Hall equation, save the lowest eigenvalue energy and use the corresponding coefficient vector to create a density function for another iteration.
    We compare the energy differences between iterations until it's less than a chosen value.

\section{QM/MM}
    In this work, we will be focused on the behavior of a molecule after an electron absorbs the energy of a photon.
    In the previous sections we have discussed how quantum mechanics can be used for chemical calculations;
    however, in many applications, the accuracy of QM is not needed and more computationally cheaper method would be more appropriate.
    For these situations many computational chemist use classical electrical force field dynamics, treating atoms as point charges.
    QM/MM was developed to manage computational costs by separating a calculation into a quantum mechanical (QM) region and a classical mechanical (MM) region.\cite{Karplus2014}
    This allows the user to have the accuracy where needed while not wasting resources on unwanted calculations such as the dynamics of water molecules far from the protein of interest.
    For the vast majority of our calculations, we will have a QM solute and a few nearby QM solvents surrounded by MM solvents.
    The Hamiltonian for this system is 
    \begin{equation}
     \mathbf{H}_{eff}=\mathbf{H}_{QM}+\mathbf{H}_{MM}+\mathbf{H}_{QM/MM} 
    \end{equation}
    with
    \begin{align}\label{eq:qmmm}
      \mathbf{H}_{QM/MM}=-\sum_{e}\sum_mq_m\mathbf{h}_{electron}(\bar{r}_e,\bar{r}_m)\\
      +\sum_q\sum_mz_qq_m\bar{\mathbf{h}}_{core}(\bar{r}_q,\bar{r}_m)\\
      +\sum_m\sum_q\left( \frac{A_{qm}}{r_{qm}^{12}}-\frac{B_{qm}}{r_{qm}^6} \right)
    \end{align}
    where $e$, $m$, and $q$, are the electron, MM atom, and QM core indices respectively;
    $q_m$ is the charge on the MM atom $m$, $z_q$ is the charge on the QM atom q, $\bar{r}$ is the coordinate vector, $r_{mq}$ is the distance between atoms $m$ and $q$ and $A$ and $B$ are the Leonard-Jones interaction parameters.\cite{Walker2008}

    \begin{multiFigure} 
    \addFigure{0.4}{../Oral/Images/qm_mm.png}
    \addFigure{0.4}{../Oral/Images/qm_mm_pme.png}
    \captionof{figure}[QM/MM Diagram]{a) single cell. b) representation of the periodic nature of the system.}
    \label{fig:QMMMDiagram}
    \end{multiFigure}

    Figure \ref{fig:QMMMDiagram} gives and example of a QM/MM systems.
    The atoms of the drawn out molecule will be described at the QM level of theory.
    The MM atoms in the volume immediately surrounding the molecular, label QMCut, will be the MM atoms included in equation \ref{eq:qmmm}.
    To simulate a solute in solvent, we treat the provided box as a cell, that is repeated infinitely many times.
    Particle Mesh Ewald calculations are then used to calculate the long distance interactions of the periodic boxes.
    This is performed by treating the charge and potential in the long range, inter box distances, as sums in Fourier space.\cite{Darden1993}
    Note that the QM region must be treated as single point charges for this calculations.
    The Mulliken charges of the current state are used for these calculations.
    Once the sums are complete, a fast Fourier transform is performed to obtain energy and forces.
    Charges from the MM region outside QMCut, will be used to provide a Particle Mesh Ewald correction to the new Fock Matrix.\cite{Walker2008}

    Long range interaction, from those outside the cutoff, considered vital for the understanding of solvent effects, are treated using SQM’s implementation of Particle Mesh Ewald.
    Trajectories use periodic boundary conditions to simulate an explicit solution, treating the system box as cells repeated infinitely many times in all directions.
    Particle Mesh Ewald calculations then determine the long-distance interactions of these periodic boxes, treating the charges and potentials in the long-range inter-box distances as sums in Fourier space treating atoms in the QM region of these calculations as Mulliken point charges.
    Once the sums are complete, SQM performs a fast Fourier transformation to obtain the long-range corrections to the energy and forces.  

    A general timestep would be as follow: 

    Calculate the MM ewald potentials using the classical charges from the MM atoms Construct the Hamiltonian matrix as if the QM region was in vacuum.
    Add the one electron terms for the interaction between QM atoms and the MM atoms within the cutoff to the Hamiltonian.
    Within the SCF routine, copy the Hamiltonian to the fock matrix, and add the two-electron integrals.
    Calculate the QM ewald potential using the iteration’s Mulliken charges, then add the ewald potentials for both QM and MM atoms to the Fock Matrix.
    The SCF procedure continues until convergence resulting in a density matrix that incorporates the presence of solvents.

\section{Configuration Interaction}
    The previous calculations resulted in a slater determinant filled with molecular orbitals that approximates the ground state.
    In order to determine the excited states, further steps must be performed.
    Steps performed after Hartree Fock, are appropriately named post Hartree Fock Methods.
    In this work we use the configuration interaction methodology.

    The Hartree fock's slater determinant, \(\Phi_0\), contains the lowest energy molecular orbitals.
    These filled orbitals are known as the occupied orbitals which will label with letters ab....
    The other available orbitals that weren't filled are considered virtual, labeled ij....

    New determinants can be made by swapping virtual and occupied orbitals.

    For example
    \begin{equation}
      \bm{\Phi}_c^i
    \end{equation}
    would be a determinant created by swapping the occupied orbital \(c\) with the virtual orbital \(i\) and
    \begin{equation}
      \bm{\Phi}_{cd}^{ij}
    \end{equation}
    would be a determinant created by swapping occupied orbitals \(c\) and \(d\) with orbitals \(i\) and \(j\).

    For K occupied orbitals, only K swaps can be made for a single determinant.
    For each molecular orbital, there are two spin states \(\alpha\) and \(\beta\) which means for K orbitals, and N electrons, there are
    \begin{equation}
      2K \choose N
    \end{equation}
    The full CI wavefunction, \(\bm{Psi}\), is linear combination of all of these determinants.
    The choose function limits the use full CI to small molecules.

    For larger molecules, the swap is limited to single, referred to as configuration interaction singles (CIS), to doubles (CID), or to both (CISD).
    For CIS, the new wavefunction can be written as

    \begin{equation}
    \bm{\Psi}_{CIS} = c_0\bm{\Phi}_0 + c_a^i\sum_i^N\sum_a^{K-N}\bm{\Phi}_a^i
    \end{equation}
    where \(c_0\) and \(\Phi_0\) are the coefficients and determinant for the Hartree Fock ground state.

    To solve for these coefficients, we use a similar method of solving an eigenvalue equation like that performed in \ref{secular}.

    \begin{equation}
      \bm{H}\vec{c} = \bm{e} \bm{S} \vec{c}
    \end{equation}
    where
    \begin{align}
      H_{ji} &= \left<\bm{\Phi}_b^j \right| \bm{H} \left| \bm{\Phi}_a^i \right>
      S_{ji} &= \left<\bm{\Phi}_b^j | \bm{\Phi}_a^i \right>
    \end{align}
    are the Hamiltonian \(\bm{H}\) and overlap \(\bm{S}\) matrices.
    When diaganolized, \(\vec{c}\) and \(\bm{e}\) are the coefficients and the energies of the CIS wave functions composed as a linear sum of the exchange determinants.

    When using CIS, the addition of the single exchange determinants have no effect on the ground state.
    Some electron correlation is accounted for excited states due to the linear combination of the mixed singly excited determinants.

\section{Adiabatic Dynamics}
	Excited-state calculations implement the Collective Electronic Oscillator (CEO) approach developed by Mukamel and coworkers, which solves the adiabatic equation of motion of a single electron density matrix.
	The single-electron density matrix is defined by  

    \begin{equation}
	\rho_{g\alpha_{nm}}t = \left< \psi_\alpha t \right| c_m^\dagger c_n \left | \psi_g t \right>
    \end{equation}

    where \(\psi_g\) and \(\psi_\alpha\) are the single-electron wave functions of the ground-state and \(\alpha\) state respectively.
    cm†(cn) is the creation(annihilation) operator summed over the atomic orbital \(m\) and \(n\), whose size is determined by the basis set.
    The basis set coefficients of these atomic orbits are calculated in the previous SCF step and account for the presence of solvents.
    The CIS approximation is applied, creating the normalization condition 

    \begin{equation}
	\sum_{n,m} (\rho_{g\alpha})^2_{n,m} = 1
    \end{equation}

    Recognizing that \(\rho_{g\alpha}\) represents the transition density from the ground to the \(\alpha\) state, we solve the Liouville equation of motion 

    \begin{equation}
	\hat{\mathcal{L}}\bm{\rho}_{0\alpha} = \Omega \bm{\rho}_0\alpha,
    \end{equation}
    with \(\mathcal{L}\) being the two-particle Liouville operator and \(\Omega\) the energy difference between the \(\alpha\) state and the ground state.

    The action of the Liouville operator can be found analytically by
    \begin{equation}
    \mathcal{L} \bm{\rho}_{o\alpha} = \left[ \bm{F}^{\vec{R}} (\bm{\rho}_{00}),\bm{\rho}_{0\alpha} \right] +
    \left[ \bm{V}^{\vec{R}} (\bm{\rho}_{0\alpha}), \bm{\rho}_{00} \right]
    \end{equation}

    where \(\bm{F}^{\vec{R}}\) is the Fock operator and \(\bm{V}^{\vec{R}}\) is teh column interchange operator.

    The diagonalization of this Liouville equation of motions uses Davidson diagonalization technique, which brings the computational costs from an otherwise O(n6) to O(n3). 

    The forces are then calculated analytically by the gradient of the ground state energy and the excited state energy. 

    \begin{equation}
    \vec{\nabla} E_\alpha = \vec{\nabla} E_0 + \vec{\nabla}\Omega_\alpha
    \end{equation}

    With the gradient of the ground state being calculated by

    \begin{equation}
    \vec{\nabla}E_0 = \frac{1}{2} \text{Tr} \bm{t}^{\vec{R}} + \bm{F}^{\vec{R}}\bm{\rho}_{00}
    \end{equation}
    and the gradient of the excited state being 
    \begin{equation}
    \vec{\nabla}\Omega_\alpha = \text{Tr} \bm{F}^{\vec{R}} \left( \bm{\rho}_{\alpha\alpha} - \bm{\rho}_{00} \right) + \text{Tr} \bm{V}^{\vec{R}} \bm{\rho}_{0\alpha}^\dagger \bm{\rho}_{0\alpha}
    \end{equation}
    where \(\rho_{ij}\) represents the density or transition density matrix for states \(i\) and \(j\),
    \(\bm{F}\) is the Fock matrix,
    \(t\) is the the kinetic operator acting on one-electron, and \(\bm{V}\) is the column interchange operator.

\section{Non-Adiabatic Dynamics}

The MDQT approach utilized in this work as a modified version of the Tully Surface Hopping method.\cite{tully2012perspective, tully1990molecular}
Here the quantum wave function is approximated using a swarm of independent trajectories.
During time steps, these trajectories propagate along adiabatic surfaces;
However, between time steps, these trajectories are allowed to transition from one state to another in a Monte Carlo like fashion.
That number oftrajectories in any given state corresponds to that state's quantum probability.

We define the Hamiltonian

\begin{equation} \label{eq:tullyHamiltonian} \mathbf{H} = \mathbf{T}(\mathbf{R}) +
  \mathbf{H}_{el}(\mathbf{r},\mathbf{R})
\end{equation}

where \(\mathbf{T}(\mathbf{R}) \) is the nuclear kinetic energy operator and \(\textbf{H}_{el}\) is the electronic Hamiltonian.

We expand the the total wavefunction, \(\Psi\) into the adiabatic state wavefunctions \(\phi\)
\begin{equation}
  \Psi(\textbf{r}, \textbf{R}, t) = \sum_j c_j(t)\phi_j(\textbf{r}; \textbf{R}) = c_j \left| \phi \right>
\end{equation}
where \(\textbf{r}\) and \(\textbf{R}\) are the electronic and nuclear coordinates respectively.


The matrix elements of the electron Hamiltonian become

\begin{equation} \label{eq:tullyVelements}
  V_{jk}(\mathbf(R))=\left<\phi_j(\mathbf{r};\mathbf{R})\right|\mathbf{H}_{0}\left.(\mathbf{r};\mathbf{R})\phi_k(\mathbf{r};\mathbf{R})\right>
\end{equation}
and the time-dependent Shrodinger equation can then be written as

\begin{equation}
  i\hbar\dot{c}_j = c_k ( V_{jk} - i\hbar \left< \phi_j | \dot{\phi}_k \right> ).
\end{equation}
The term \(\left< \phi_j | \dot{\phi}_k \right>\) represents the coupling between the jth and kth state and is most commonly referred to as the nonadiabatic coupling term.

At each step we perform a montecarlo like decision

\begin{equation} \label{eq:tullyjump2} 
\sum_{j=1}^{k-1}g_{ij} < \zeta  \le \sum_{j=1}^{k}g_{ij}
\end{equation}
hopping from state i to k when
\begin{equation} \label{eq:tullyjump1} 
  \zeta < g_{ik}
\end{equation}
where \(\zeta\) is a uniformly distributed random number from 0 to 1, and

\begin{equation}
g_{ik} = \frac{b_{ki}(t=0)\Delta t}{a_{ii}(t=0)}
\end{equation}

with

\begin{equation} \label{eq:tullyb2a} 
b_{kj} =
        \frac{2}{\hbar}\Im\left(a_{kj}^*V_{kj}\right) - 2\Re\left(a_{kj}^*
         \dot{\mathbf{R}} \cdot \mathbf{d}_{kj}\right).
\end{equation}
$a_{kj}$ are the off diagonals of the density matrix $a_{kj} = c_k^* c_j$ and
$\mathbf{d}_{kj}$ is the non-adiabatic coupling vector

\begin{equation} \label{eq:tullynacoupling} 
\mathbf{d}_{kj}\mathbf(R) =
  \left<\phi_{k}(\mathbf{r};\mathbf{R})\right|\mathbf{\nabla}_{\mathbf{R}}\left.\phi_j(\mathbf{r};\mathbf{R})\right>.
\end{equation}
We use the Collect Oscillator Approach to calculate the non-adiabatic coupling terms \(\mathbf{R} \cdot \mathbf{d}_{kj}\) ``on the
fly''. \cite{tommasini2001electronic, tretiak1996collective, tretiak2009representation, chernyak2000density}

Inconsistencies arise from solely using the Tully surface hopping approach.
Trajectories transfer between  the various adiabatic potential energy surfaces instantaneously based off the QM state coefficients.
These coefficients are determined using the integral of the TDSE on multiple trajectories.
Each trajectory if unmodified will keep in phase even after spatial separation.
Furthermore, if dealing with a system with a dense electronic state structure, its likely that the ordering of these states will switch during general dynamics.
We apply a dechohence correction as well as a trivial crossing accounting system as performed in previous research.
% Modified from old template.
\chapter{Implementation Details} \label{implementationDetails}

\section{Introduction to AMBER and NEXMD}

NEXMD, currently being developed by the Tretiak lab in los Alamos, has a proven track record of performance on the stimulation of ultra-fast non-adiabatic behaviors.
It’s ability to solve state coupling equations on-the-fly has found great utility for systems with hundreds of atoms.
Numerous studies have implemented the method for research into topics including the study of chlorophyll organic conjugated molecules, and pi conjugated macrocycles. \cite{zheng2017photoinduced,nelson2014nonadiabatic,alfonso2016interference,wu2006exciton,Ondarse-Alvarez2016} 
Such studies with NEXMD have been limited to implicit solvents.
No method to provide NEXMD with QM/MM capabilities have yet to be implemented.

Amber is primarily known as a classical force-field molecular dynamics package.
It’s a massive project maintained by people across the globe that's been designed to work with very large systems ranging in the tens of thousands of atoms. \cite{case2020a}
Amber is capable of a huge range of simulations from replica exchange to study ph-dependent conformation chagnges to QM/MM umbrella sampling using nudge elastic bands. \cite{cruzeiro2020exploring, ghoreishi2019fast,sarkar2019ph}
Most importantly for this research, it has a proven track record of doing QM/MM solvent-solute simulations using periodic boundary conditions.

\section{Schematics}
\noindent
\begin{minipage}[c]{\textwidth}
  \centering
  \includegraphics[width=0.5\linewidth]{../Paper2/scripted_diagrams/nasqm_overview.png}
  \captionof{figure}{Swim-lane diagram describing the common timestep of the SANDER-NEXMD interface.}
  \label{scheme:nasqm}
\end{minipage}\bigskip

The swim-lane chart in figure \ref{scheme:nasqm} describes a common time-step that occurs within the SANDER-NEXMD interface.
First users initiate the program through SANDER, a program found in AMBERTOOLS. SANDER uses NEXMD to calculate the energies and forces of the QM atoms, check for trivial crossings, and propagate the quantum coefficients.
With these results, SANDER performs the QM/MM procedures to derive the accelerations and velocities for the classical time step.
NEXMD then decides whether to perform a state transitions, adjusting the velocities as needed.
Finally SANDER propagates the nuclear coordinates and the cycle continues for the rest of the dynamics.

When users initiate SANDER, they're provide the usual SANDER inputs of a
coordinate, parameter, and sander control files.
In addition they will include a file specific to
NEXMD which describes the QM and Non-adiabatic behavior.
This interface, incorporates SANDER's implementation of QM/MM as described in previous literature to generate a solvent inclusive ground state density matrix utilized by NEXMD's excited state calculations.
Sander controls the interactions between the QM and MM regions.


SANDER calls NEXMD providing the function calls with the QM coordinates, MM charges, and Langevin thermostat parameters.\cite{paterlini1998constant}
NEXMD calculates the energies of the QM atoms with electrostatic interactions from the MM point charges using CIS, TDHF, or TDDFT.
A variety of Hamiltonians are available; however, AM1 has been shown to provide very reasonable computational cost to accuracy for our systems of interest.\cite{silva2010benchmark}
An analysis of parameter choices can be found in previous literature.\cite{nelson2012nonadiabatic}

We use SANDER's QM/MM implementation to provide approximations of the solvent interactions.\cite{Walker2008}
   SANDER's combined QM/MM Hamiltonian represents MM atoms as point charges and QM atoms as electronic wave-functions.
   The effective Hamiltonian uses the aforementioned hybrid approach
   \begin{equation}
     \mathbf{H}_{eff} = \mathbf{H}_{QM} + \mathbf{H}_{MM} + \mathbf{H}_{QM/MM}
   \end{equation}
   where \(\mathbf{H}_{QM}\), \(\mathbf{H}_{MM}\), \(\mathbf{H}_{QM/MM}\) are the Hamiltionians for the QM to QM, MM to MM, and QM to MM hybrid interactions.
   \(\mathbf{H}_{MM}\) is not considered during the electronic calculations due to it independence from the electronic distribution.
   \(\mathbf{H}_{QM}\) is the electronic Hamiltonian used in vacuum QM SCF calculations.
   \(\mathbf{H}_{QM/MM}\) represents the interactions between the QM charge density and MM atoms treated as point charges.
   For computational efficiency we limit the range of this interaction by a cuttoff, set by the user, generally in the range of 10-16 \(\AA\) from the paremeter QM atoms.
   For short range interactions the hybrid \(\mathbf{H}_{QM/MM}\) can be expanded into

   \begin{align}
     \mathbf{H}_{QM/MM} = &- \sum_i \sum_m q_m \hat{h}_{electron} (\vec{r}_i,  \vec{r}_m)\\
			  &+ \sum_q \sum_m q_q q_m \hat{h}_{core} (\vec{r}_q, \vec{r}_m)\\
			  &+ \sum_m \sum_q \left( \frac{A_{qm}}{r_{qm}^{12}} - \frac{B_{qm}}{r_{qm}^6} \right)
   \end{align}
   where \(i\) is the electron, \(m\) the MM atom, and \(q\) the combined nuclei and core electrons of the QM atoms.
   A and B are the Lennard-Jones interaction parameters where \(r_{qm}\) is the distance between the MM and QM atoms.
   \(q\) is the charge and \(r\) is the ccoordinate vector.
   \(\hat{h}_{core}\) represents the electronic interactions between the MM charges and the core of the QM atoms.
   \(\hat{h}_{electron}\) represents the interactions between the MM charges and either the charge density of the QM region when using Semi-emprical hamiltonians, or by using the Mulliken charges in the case of DFT.

   Long range interaction, from those outside the cutoff, considered vital for the understanding of solvent effects, are treated using SQM’s implementation of Particle Mesh Ewald.
   Trajectories use periodic boundary conditions to simulate an explicit solution, treating the system box as cells repeated infinitely many times in all directions.
   Particle Mesh Ewald calculations then determine the long-distance interactions of these periodic boxes, treating the charges and potentials in the long-range inter-box distances as sums in Fourier space treating atoms in the QM region of these calculations as Mulliken point charges.\cite{essman1995smooth}
   Once the sums are complete, SQM performs a fast Fourier transformation to obtain the long-range corrections to the energy and forces. 


   For a general timestep QM/MM interaction will be added to the density matrix as follows: 
\begin{enumerate}
   \item Calculate the MM ewald potentials using the classical charges from the MM atoms.
   \item Construct the Hamiltonian matrix as if the QM region was in vacuum.
   \item Add the one electron terms for the interaction between QM atoms and the MM atoms within the cutoff to the Hamiltonian.
   \item Within the SCF routine, copy the Hamiltonian to the fock matrix, and add the QM/MM two-electron integrals.
   \item Calculate the QM ewald potential using the iteration’s Mulliken charges, then add the ewald potentials for both QM and MM atoms to the Fock Matrix.
   \item The SCF procedure continues until convergence resulting in a density matrix that incorporates the presence of solvents. 
\end{enumerate}

Excited-state calculations implement the Collective Electronic Oscillator (CEO) approach developed by Mukamel and coworkers, which solves the adiabatic equation of motion of a single electron density matrix.\cite{Mukamel1997,tretiak02_densit_matrix_analy_simul_elect}
The single-electron density matrix is defined by
\begin{equation}
  (\rho_{g\alpha})_{nm} = \left< \psi_{\alpha} (t) \right| c_m^{\dagger} c_n \left| \psi_{g} (t) \right>
\end{equation}
where \(\psi_g\) and \(\psi_\alpha\) are the single-electron wave functions of the ground-state and \(\alpha\) state, and \(c_m^{\dagger} c_n\) is the creation(annihilation) operator summed over the atomic orbital \(m\) and \(n\), whose size is determined by the basis set.
The basis set coefficients of these atomic orbits are calculated in the previous SCF step and account for the presence of solvents.
The CIS approximation is applied, creating the normalization condition 

\begin{equation}
  \sum_{n,m} (\rho_{g\alpha})^2 = 1
\end{equation}

Recognizing that \(\rho_{g\alpha}\) represents the transition density from the ground to the \(\alpha\) state, we solve the Liouville equation of motion 

\begin{align*}
  \hat{\mathbf{\mathcal{L}}}\mathbf{\rho}_{0\alpha} = \Omega\mathbf{\rho}_{0\alpha}
\end{align*}


with \(\mathbf{\mathcal{L}}\) being the two-particle Louiville operator and
\(\Omega\) the energy difference between the \(\alpha\) state and the ground
state. \(\rho_{0\alpha}\) is the single- electron density matrix

\begin{align*}
  (\mathbf{\rho}_{0\alpha})_{nm} =  \left< \psi_{\alpha} \right| c_m^\dagger c_n \left| \psi_0 \right>.
\end{align*}
where \(\psi_{g}\) and \(\psi_{\alpha}\) are the single-electron wavefunctions for the groundstate and singly excited state \(\alpha\), and
\(c_m^\dagger\) and \(c_n\) are the creation and annihilation operator respectively. The coefficients for the atomic orbital basit sets are derived prior using the QM/MM methodology from the SQM package.\cite{Walker2008} The Liouville operator can be found analytically using 

\begin{align*}
  \hat{\mathbf{\mathcal{L}}}\mathbf{\rho}_{0\alpha} = \left[ \mathbf{F}^{\vec{R}}(\mathbf{\rho}_{00}), \mathbf{\rho}_{0\alpha} \right]
  + \left[ \mathbf{V}^{\vec{R}}(\mathbf{\rho}_{0\alpha}), \mathbf{\rho}_{00} \right]
\end{align*}

where \(\textbf{F}^{\vec{R}}\) is the Fock operator and \(\textbf{V}^{\vec{R}}\) is the column interchange operator.

The forces are also calculated analytically using the gradient of the ground state
energy and the excited state energy.

\begin{align*}
  \vec{\nabla} E_{\alpha} = \vec{\nabla} E_0 + \vec{\nabla} \Omega_{\alpha}
\end{align*}

With the gradient of the ground state being calculated by

\begin{align*}
  \vec{\nabla} E_0 = \frac{1}{2}\mathbf{Tr} \left[ \left( \mathbf{t}^{\vec{R}} + \mathbf{F}^{\vec{R}} \right) \mathbf{\rho}_{00} \right]
\end{align*}
where \(\textbf{t}\) is the single-electron kinetic operator.

The gradient of the excited state can derived using

\begin{align*}
  \vec{\nabla}\Omega_{\alpha} = \mathbf{Tr} \left[ \mathbf{F}^{\vec{R}} (\mathbf{\rho}_{\alpha\alpha} - \mathbf{\rho}_{00})\right]
  + \mathbf{Tr} \left[\mathbf{V}^{\vec{R}}\mathbf{\rho}_{0\alpha}^\dagger \mathbf{\rho}_{0\alpha} \right]
\end{align*}

where \(\mathbf{\rho}_{ij}\) represents the density or transition density matrix for states i and
j, \(\mathbf{F}\) is the Fock matrix and \(\mathbf{V}\) is the column interchange operator.

\begin{table}[H]
      \caption{Benchmark average timings for 0 to 1 excited states}
      \label{table:benchmarks1}
      % \begin{center}
      \begin{tabularx}{\textwidth}{XXXX}\hline
N States & N Solvent & Adiabatic (ms/step) & NonAdiatic (ms/step)\\\hline
0        & 0         &  1.34 & 2.89\\
0        & 1         &  2.35 & 4.32\\
0        & 3         &  3.14 & 8.85\\
0        & 5         &  5.89 & 10.42\\
0        & 10        & 11.87 & 13.32\\
0        & 15        & 16.34 & 19.81\\
1        & 0         &  1.34 & 25.22\\
1        & 1         &  2.35 & 29.12\\
1        & 3         &  3.14 & 32.98\\
1        & 5         &  5.89 & 36.01\\
1        & 10        & 11.87 & 40.33\\
1        & 15        & 16.34 & 45.63\\\hline
      \end{tabularx}
    \end{table}

    \begin{table}[H]
      \caption{Benchmark timings for 3 5, and 10 excited states}
      \label{table:benchmarks2}
      % \begin{center}
      \begin{tabularx}{\textwidth}{XXXX}\hline
N States & N Solvent & Adiabatic (ms/step) & NonAdiatic (ms/step)\\\hline
3        & 0         &  2.34 & 4.89\\
3        & 1         &  3.35 & 8.32\\
3        & 3         &  4.14 & 15.85\\
3        & 5         &  7.89 & 20.42\\
3        & 10        & 14.87 & 25.32\\
3        & 15        & 19.34 & 38.81\\
5        & 0         &  2.34 & 50.22\\
5        & 1         &  3.35 & 58.12\\
5        & 3         &  4.14 & 64.98\\
5        & 5         &  7.89 & 72.01\\
5        & 10        & 13.87 & 80.33\\
5        & 15        & 20.34 & 90.63\\
10        & 0         &  4.34 & 100.22\\
10        & 1         &  6.35 & 120.12\\
10        & 3         &  8.14 & 128.98\\
10        & 5         &  14.89 & 144.01\\
10        & 10        & 26.87 & 160.33\\
10        & 15        & 40.34 & 180.63\\\hline
      \end{tabularx}
    \end{table}

Figures \ref{table:benchmarks1} and \ref{table:benchmarks2} show the timings for the Sander-NEXMD interfaces. 
Benchmarks consisted of a single PPV\(_3\)-NO\(_2\) molecule surrounded by 3423 CCl\(_4\) molecules.
PPV\(_3\)-NO\(_2\) consistes of 50 atoms, and every additional CCL\l\(_4\) molecule adds an additional 5 atoms.
A various number of solvent molecules are included in the QM region as shown in the second column.
The adiabatic dynamics seems to increase on the order of O(3) for the number of solvents and O(1) for the number of states.
This coincides to what we expect from the precedures found in configuration interactions.
The order for the number solvents orresponds to what we expect using AM1 semi-empirical method.
For non-adiabatic dynamics, the computation time seems to increase on the order of O(3) for the number of solvents and O(2) for the number of states.
The computational order on the number of solvents does is near identical for to the adiabatic since the non-adiabatic dynamics perform the same AM1 calculations.
The time order in the number of states however, differs because of the time used to calculate the non-adiabatic coupling vector and terms.
% Modified from old template.
\chapter{Spectroscopic Analysis of PPV\(_3\)-NO\(_2\)}

\section{Introduction}
   NEXMD is an efficient program for the simulation of photoinduced dynamics of extended conjugated molecular systems involving manifolds of coupled electronic excited states over timescales extending up to 10s of picosectonds. \cite{sifain2018photoexcited, Bjorgaard2015, case2020a, tretiak02_densit_matrix_analy_simul_elect, malone2020nexmd}
   It includes solvent effects using implicit models. These implicit solvents provide insight into the electrostatic forces, but the use of explicit solvent should provide additional information about the quantum or steric effects , allowing the simulation of electron transfer processes due to the stabilization of charge separation in the excited state.
   Many multichrormophoric molecular systems are soluble in polar solvents such as water, where such simulations could provide sought after insight into effects of charged side groups on the structural sampling, structural rearrangement, and transition density redistribution during electronic relaxation.  

   Adequate sampling of the solvent and solute configuration space, including hundreds to thousands of solvent molecules, currently cannot be achieved by QM calculations alone due to the infeasible computational cost of such extensive systems. \cite{barbatti2011nonadiabatic}
   QM/MM methods exist to allow the computationally expensive QM calculations to be performed only on the area of interest while propagating the rest of the system with cheaper classical dynamics.
   The SANDER program in the AMBER molecular dynamics package performs classical molecular dynamics with the use of periodic boundary conditions with high optimization.
   SANDER with QM/MM performs well with systems with tens of thousands of MM atoms and currently calculates QM/MM with semi-empirical Hamiltonians and DFT; however, previous versions did not provide options for excited-state MD.  

   In this work, we redirect AMBER's SANDER package from its usual semi-empirical QM package, SQM, to a modified NEXMD library.
   SANDER linked to the NEXMD library performs adiabatic dynamics at ground-state and CIS excited-state potential energy surfaces on ~100 atoms at QM and 1000s at MM.
   We apply our method to an analysis of a three-ring para-phenylene vinylene oligomer (PPV3-NO2).
   We look at how explicit solvents affect PPV3-NO2's excited state structure as well as its absorption and emission spectra.

\section{Simulation Methods}

\noindent
\begin{minipage}[c]{\textwidth}
  \centering
  \includegraphics[width=5in]{../Paper1/Images/trajectory_diagram/trajectories.pdf}
  \captionof{figure}[Description of Adiabatic Dynamics]{Description of the Adiabatic Simulation}
  \label{fig:adiabaticDynamics}
\end{minipage}\bigskip

Figure \ref{fig:adiabaticDynamics} visually describes the layout of our simulations performed in vacuum, methanol, choroform, and carbontetrachloride.
First, a molecular dynamics simulation of 320 ps was run in the ground state at 300K (NVT) using the General AMBER Force Field ground state Hamiltonian and periodic boundary conditions, as guided form previous convergence studies. \cite{silva2010benchmark}
128 evenly spaced snapshots were collected from this simulation and used as intial conditions for another 128 individual 10ps-simulations equilibrated at 300K (NVT) using QM/MM ground state Hamiltonian.
QM calculatations were performed using configuration interaction singles (CIS) with the AM1 hamiltonian which has previously been show to provide reasonable accuracy for computational cost. \cite{silva2010benchmark}
QM/MM excited state molecular dynamics simulations were run during 10 ps starting from the final configuration of each of the 128 ground state QM/MM trajectores by vertical excitation to the S1 state. \cite{nelson2012nonadiabatic}
A classical time step of 0.5 fs and a Langevin thermostat with friction constant of 2 ps-1 have been used either for ground state and excited state QMMM simulations. \cite{nelson2012nonadiabatic}
The CIS calculations include the first of excited states of the PPV3-NO2 molecule along with the stated number of solvent molecules closest to the central benzene ring.
We treat all other solvent molecules using classical dynamics.
We restrict the QM solvents from drifting away from the solute and the other MM solvents from drifting closer than these QM solvents.

As shown in Figure, the excited-state density resides towards the center of the molecule on the vinyl groups nearest to the phenyl group.
Charge movements on solvents far from this concentration of density cause negligible energy differences.
To maximize the utility of the QM/MM calculation, we only include the solute and the solvent molecules nearest to this central phenyl group in the QM calculations.
To prevent these solvents from drifting during the trajectories, we implement a simple harmonic restraint using AMBER's NMR restraints.
We select the N solvent molecules based on the proximity of the solvent atom closest to any atom located on the central benzene ring.
We then restrain these solvents using the harmonic constraint on the distance from the center of geometry of the solvent to the center of geometry of the central phenyl group.
We also restrain the solvent molecules not included in the QM calculation from getting closer to the central phenyl than the QM solvents, effectively making a spherical shield around the solute's central phenyl group.
Since the distance between the center of geometries and the closest atoms are not necessarily equal, solvent atoms could initially be pushed either inside or outside this shield during the transition from MM to QM.
However, this push only occurs during the initial equilibration of the QM ground-state calculations excluded from any analysis.
Once the QM calculations begin, these constraints persist throughout all further calculations.
For our CCl4 simulations with the 5 nearest solvents included (CCl4-5QM), the restriction barrier had an average radius of 6.28 \AA with the origin at the center of the central phenyl group.

\section{Results}

\subsection{Bond Length Alternation}
	The structural differences between the excited-state and ground-state of PPV3-molecules are presented clearly by distortions in the C=C and C-C bonds found in the vinylene segment.\cite{tretiak02_densit_matrix_analy_simul_elect, karabunarliev2000rigorous, karabunarliev2000adiabatic, nelson2014nonadiabatic}
	These distortions can be measure by bond length alternation (BLA)

	\begin{equation}
	\frac{d_i + d_e}{2} - d_e,
	\end{equation}

	where \(d_i\) and \(d_e\) are the interior and exterior bonds, and \(d_c\) is the central bond.
	This value represents the differences between the double and single bonds of the vinylene sections.
	The BLA is a descriptor for \(\pi\) bond distributions. \cite{tretiak2002conformational}
	In this system, we analyze the BLAs of two separate bond sets, the bonds d1-3 (near-set) and d4-6 (far-set) seen in Scheme 2. 

	\noindent
	\begin{minipage}[c]{\textwidth} 
	  \centering
	  \includegraphics[width=5in]{../Paper1/Images/bla.png}
	  \captionof{figure}[BLA of bonds during adiabatic dynamics]{BLA of bonds d1-3 (left) and d4-6 (right) during the last 4 ps of S0 dynamics and 10 ps of S1 dynamics.
	    QM energy and force calculations include the 20 solvents nearest the central ring.}
	  \label{fig:bla_adiabatic}
	\end{minipage}\bigskip



\begin{table}[H]
  \caption[Adiabatic Bond Length Alternation]{Bond Length Alternation summary for PPV\(_3\)-NO\(_2\) in various solvents with 20 solvents included in the QM region.}
  % \begin{center}
  \begin{tabularx}{\textwidth}{XXXXXXXXX}\hline
    Molecule   & d\(_1\)  & d\(_2\) & d\(_3\) & BLA\(_{\textbf{near}}\) & d\(_1\)  & d\(_2\) & d\(_3\) & BLA\(_{\textbf{far}}\)\\\hline
    Vacuum     & 1.429     & 1.375    & 1.418    & 0.049              & 1.441     & 1.365    & 1.427   & 0.069\\
    CCl\(_4\)  & 1.427     & 1.376    & 1.417    & 0.046              & 1.441     & 1.365    & 1.426   & 0.068\\
    CH\(_3\)OH & 1.422     & 1.382    & 1.415    & 0.037              & 1.444     & 1.362    & 1.431   & 0.076\\
    CHCl\(_3\) & 1.423     & 1.380    & 1.415    & 0.039              & 1.443     & 1.365    & 1.429   & 0.074\\\hline
  \end{tabularx}
\end{table}


	Figure \ref{fig:bla_adiabatic} show the bond length alternation of PPV3-NO2 in various solvents during the S1 trajectories where the dashed lines represent equilibrated ground state values which are near 0.110 Å for sets d1-3 and d4-6 regardless of solvent enviornment.
	Within the first couple hundred femtoseconds after the excitation to S1, the central bonds expand, while the interior and exterior bonds contract.
	For all solvent environments, this bond restructuring is amplified by close proximity to the amino group, where we see an average drop of 0.07 Å in sets d1-3 compared to a 0.04 Å drop in sets d4-6.
	The strength of this amplification is dependent on the solvent environment where the BLA difference between the far and near sets PP3-NO2 in methanol, 0.034 Å, surpases that found in carbon tetrachloride, 0.022 Å.

	Table 2 presents further details of the S1 BLA simulation.
	In all cases, the exterior bond (d1 and d4) becomes slightly longer than the interior bond (d3 and d6).
	The BLAs from the near and far sets therefore split.
	In the near set, the S0 and S1 BLA converge to 0.1091 Å and 0.00453 Å, respectively.
	In the far set, these numbers are 0.1103 Å and 0.0697 Å, respectively.
	The smaller bond length spread in the near set, along with the lower BLA, suggests more delocalization on those bonds than in the far set. 

	The number of solvents included in the QM calculations had little effect on the convergence of the distances or BLA.
	PPV3-NO2 had similar ground state BLAs of around 0.11 Å for both near and far sets matching results on similar systems regardless of the solvent. \cite{nelson2011nonadiabatic}
	The information presented is averaged over time after relaxation across all trajectories.
	The S1 BLAs varied between the solvents and with the distance away from the NO2 group.
	Among the selected solvents, CH3OH has the smallest near set S1 BLA and largest far set S1 BLA.
	CCl4 has the largest near set S1 BLA, and also the smallest far set S1 BLA and has close to vacuum-like behavior.
	The CH3OH solvent seems to enlarge the BLA changes from the ground to excited states.

\subsection{Wiberg Bond Orders}
\begin{minipage}[c]{\textwidth}
\centering
\includegraphics[width=5in]{../Paper1/Images/ccl4-5s-widberg.png}
\captionof{figure}[Wiberg Bond Orders During Adiabatic Dynamics.]{Wiberg Bond Orders for PPV\(_3\)-NO\(_2\) in CCL\(_4\) with 5QM Solvents During Adiabatic Dynamics. }
\label{fig:bondOrdersAdiabatic}
\end{minipage}\bigskip


    The significant effects of S0-S1 transitions on the Cartesian measurement of BLA encourages the analysis of these bonds' quantum mechanical behavior.
    Because the double bonds elongated and the single-bonds contracted, we expect the single-bonds to gain a partial double-bond character and vice versa.
    Simple bond ordering does not consider these subtleties of a quantum electronic wave-function.
    Fortunately, the quantum mechanical descriptor, Wiberg bond index, provides a reasonable analogy of the classical Lewis structure a chemist would expect.
    Wiberg bond indexes are calculated from the density matrix by 

    \begin{equation}
    W_{AB} = \sum_{\mu\in A}\sum_{\nu \in B} | D_{\mu\nu} |^2
    \end{equation}

    where \(A\) and \(B\) are indexes of the two atoms, \(\mu\) and \(\nu\) are the atomic orbitals, and \(D\) is the density matrix.
    The method sums the electron density shared by both atoms.
    If an electron if fully localized on a single atom, the sum of the elements equals zero providing a value that matches our intuition of a bond. 

    As the bond order increases, we expect the bond to become more rigid and the bond length to shrink.
    Figure \ref{fig:bondOrdersAdiabatic} displays the bond order of bonds d1-6 for PPV3-NO2 in CCl4 with 5 QM solvent molecules.
    At time t=0, the system instantaneously transitions to the first excited state, S1.
    The Wiberg bond index then uses the density matrix for S1, leading to a sudden shift in its value.
    At S1, the bond orders of d2 and d5 instantaneously drop, and expansion of their bond lengths soon follows.
    The larger shifts in the near set correspond to the information found in the BLA analysis.
    The interior bonds, d1 and d4, have lower bond indexes than their exterior counterparts, d3 and d6.

\subsection{Torsional Angles}

\noindent
\begin{minipage}[c]{\textwidth}
  \centering
  \includegraphics[width=4in]{../Paper1/Images/dihedrals.png}
  \captionof{figure}[Tortional Angles during Adiabatic Dynamics]{Tortional angle around d1-d3, near, and d4-d6, far, in S1 within CCl\(_4\)-5QM}
  \label{dihedralAdiabatic}
\end{minipage}\bigskip

\begin{table}[H]
  \caption[]{}
  % \begin{center}
  \begin{tabularx}{\textwidth}{XXXXX}\hline
    Molecule    & S0 Near  & S1 Near & S0 Far & S0 Far\\\hline
    Vacuum      & 28.0 \(\pm\) 1.0\(^\circ\) & 12.5 \(\pm\) 0.5\(^\circ\) & 28.5 \(\pm\) 0.9\(^\circ\) & 17.4 \(\pm\) 0.8\(^\circ\)\\
    CHCl\(_4\)  & 25.3 \(\pm\) 1.2\(^\circ\)  & 12.3 \(\pm\) 0.6\(^\circ\) & 25.1 \(\pm\) 1.3\(^\circ\) & 14.7 \(\pm\) 0.8\(^\circ\)\\
    CH\(_3\)OH  & 26.6 \(\pm\) 0.9\(^\circ\)  & 11.5 \(\pm\) 0.5\(^\circ\) & 28.2 \(\pm\) 0.9\(^\circ\) & 16.2 \(\pm\) 0.8\(^\circ\)\\
    CHCl\(_3\)  & 26.8 \(\pm\) 1.1\(^\circ\)  & 11.8 \(\pm\) 0.5\(^\circ\) & 27.8 \(\pm\) 1.2\(^\circ\) & 16.0 \(\pm\) 0.7\(^\circ\)\\\hline
  \end{tabularx}
\end{table}

    We use the torsion angle around the vinylene segments as the slow nuclear coordinates of PPV3-NO2-molecules following precendent. \cite{Clark2012}
    In PPV3-NO2 systems, the excitation to S1 leads to relaxation towards a nearly planar structure Torsion angle around d1,d3 and d4,d6 are averaged over 128 trajectories to produce the near and far torsion angle data respectively. 

    For our CCl4-5QM example, the dihedral angle around the near set d1 to d2 equilibrates around 23° and 12° in the S0 and S1 states, respectively.
    For the near set, d4 to d6, these values are 23° and 15°.
    Once again, only a noticeable difference in S1.
    The time constants for the S1 dihedral relaxations are around 0.8 ps.
    Relaxation of the dihedral angles occurs by four ps. 

    Table 5 shows a summary of the torsion angles analysis of all tested solvents after five ps of relaxation after the jump to the first excited state.
    The trajectories include 20 solvent molecules within the QM calculations.
    A noticeable shift towards a planar geometry occurs in all solvents.
    This shift is greatest near the nitro group.

\subsection{Potential Energy Relaxation}

\noindent
\begin{minipage}[c]{\textwidth}
  \centering
  \includegraphics[width=5in]{../Paper1/Images/energies.png}
  \captionof{figure}[Potential Energy Relaation During Adiabatic Dynamics]{Potential energy difference from ground state for PPV\(_3\)NO\(_2\) in CCl\(_4\) with 5 QM solvent molecules.}
  \label{fig:energiesAdiabatic}
\end{minipage}\bigskip

 The absorption and the fluorescence properties are judged primarily through the difference between the ground state (S0) and the first excited state (S1) energies.
 The system starts at the S0, where it remains near the bottom of the energy well.
 Figure \ref{fig:energiesAdiabatic} shows the energies for states S0 and S1 averaged over 128 trajectories for PPV3-NO2 in CCL4 with five solvent molecules included in the QM calculations.
 During the first six ps, the system runs on the ground-state S0, and the S0 energies stay near the minimum with small oscillations caused by temperature.
 At the time 0 ps, the system instantaneously hops to the S1 potential energy surface.
 The average energy difference at t=0 between S0 and S1 is 2.93 eV, and it corresponds reasonably well with the peak of the absorption spectrum.
 When the system relaxes on the new surface, the S1 and S0 energies decrease and increase respectively, until the difference between the two is 2.50 eV agreeing with the peak found in the fluorescence spectra.
 Table ? presents the fitted decay of S1 energies using   

\begin{equation}
E = E_d \text{e}^{-t/\tau} + c
\end{equation}
where \(E_d\) is the relaxation energy drop, \(\tau\) the time constant, and c the steady state energy at S1.

\subsection{Spectra}

\noindent
\begin{multiFigure} 
  \addFigure{0.4}{../Paper1/Images/nquant_abs_comparison.png}
  \addFigure{0.4}{../Paper1/Images/nquant_flu_comparison.png}
  \addFigure{0.4}{../Paper1/Images/spectra_abs_compared.png}
  \addFigure{0.4}{../Paper1/Images/spectra_flu_compared.png}
  \captionof{figure}[Fluorescence and Absorption Spectra]{PPV\(_3\)-NO\(_2\) absorption and fluorescence spectra A-B) in CCL\(_4\) with varying number of QM solvents. C-D in various solvents with 20 included in QM region.}
  \label{fig:}
\end{multiFigure}\bigskip


    Energies, coordinates, and dipole information are acquired every ten steps.
    Equilibration times from either MM to QM state or from the S0 to S1 state range from 2-4 ps.
    We exclude the first four ps of each trajectory in the calculation of the spectra in data analysis for the absorption and emission analysis. 

    Previous studies have analyzed solvatochromic shifts in conjugated substituted PPV3-NO2 molecules with the NEXMD program in implicit solvents.
    Results with NEXMD by TD-AM1 were redshifted from the experimental results, while single-point calculations using TD-CAM-B3LYP provided by G09 in the same implicit solvent were blue shifted.
    Other NEXMD computations have shown comparable redshifts in spectra of similar molecules in implicit solvents compared to experiment. \cite{Bjorgaard2015}
    We performed similar calculations in this paper; however, in explicit solvent.
    We compare the results to those presented in implicit solvent.

    We collect the vertical excitation dipoles and oscillator strengths between the ground state S0 and S1 every five fs during the steady-state of each trajectory to produce the absorption/emission spectra of PPV3-NO2.
    We sum over excitation states averaged over the geometries and broaden the spectra using a Gaussian bin function with FWHM=0.16 eV corresponding to a 100 fs FWHM laser excitation.
    We normalized it such that the maximum absorption is 1. 

    Figure ? presents the absorption spectra for PPV3-NO2 in select solvents and vacuum.
    The shown absorption has contributions from the nine lowest energy excited states, though the S1 state is the primary contributor to the spectra.
    We found the number of solvent molecules included in the QM region caused only minor deviations in the spectra, with the largest variance (0.02~eV) occurring between the 20QM and MM CCL4 systems, as such, in figure 12, we only present results from trajectories with 20 QM solvents.
    All solvent results are redshifted from those in vacuum matching findings in previous works.
    The absorbance within methanol and chloroform were very similar, with a peak shift from vacuum of -0.04 eV.
    Within carbon tetrachloride, this shift increases slightly more to-0.06 eV. 

    Aligning with previously reported results, the fluorescence calculations found in figure 13 show an overall more intense redshift from vacuum, along with a more significant dispersion among the solvents. \cite{Park2013}
    The smallest shift, at -0.06 eV, occurs in carbon tetrachloride, while the largest, at -0.12 eV, occurs in methanol.
    Previous works have demonstrated that the energy levels of PPV3-NO2 are further stabilized by more polar solvents, a feature clearly seen by our results.

The Non-adiabatic Excited State Molecular Dynamics package (NEXMD) that SANDER is linked to for the excited state calculations done in this paper is designed to perform molecular dynamics simulations outside the Born-Oppenheimer approximations using the FSSH in implicit solvents.
The program has been tested well and is helpful in gaining insights into the dynamics of molecules prone to photo-excitations.
However, simulations in explicit solvents have been shown to produce qualitatively different results than in implicit solvent mediums.
By linking AMBER's SANDER to NEXMD, we produced a quick and efficient tool to simulate explicit solvent behavior on adiabatic excited state dynamics.
% Modified from old template.
\chapter{QM/MM Non-adiabatic Dynamics of PPV\(_3\)-NO\(_2\)}

\section{Introduction}

\section{Simulation Methods}

\noindent
\begin{minipage}[c]{\textwidth}
  \centering
  \includegraphics[width=5in]{../Paper2/scripted_diagrams/simulations-1.png}
  \captionof{figure}{Diagram of the Nonadiabatic dyanmics simulation.}
  \label{fig:nonadiabaticSimulation}
\end{minipage}\bigskip

We equilibrated the system to a temperature set to 300K. To collect a broad
enough sampling, we sampled from a 1024 ps, with a 0.5 fs timestep fully
classical trajectories using the AMBER force field. We performed a separate
trajectory for each situation combination of solute / with solvent including
whether the solvent was included in the QM calculations. We had a total of 6
separate 1024 ps classical trajectories, PPV3 in Vacuum, CH\(_3\)OH, and 5QM CH\(_3\)OH
and PPV\(_3\)-NO\(_2\) in Vacuum, CH\(_3\)OH, and 5QM CH\(_3\)OH. 1024 snapshots where taken at
1ps, 2ps .. 1024ps. We used the final frame of those tranjectories as the
initial conditions for an additional 4ps using the AM1 semiempical Hamiltonian
Born-Oppenheimer on the molecules to be included in future QM calculations to
allow the system to relax. The 4 ps timescale was determined using the
information form the previous paper. The simulations were described the Langevin
equations at a temperature set to 300 K with the Langevin friction parameter set
to 2 ps\(^{-1}\). The final frames of these QM trajectories were then used as the
initial conditions for the following pulse pump calculations.

Pump-Probe Spectroscopy is an experimental technique commonly performed in the
study of ultrafast electonic statte dynamics. In the case of conjugated polymers
in can be used to study the localized excictronic transitions that are
accessible through an excitation from the S1 state but not the ground state S0.
To simulate this behavior, we take the final snapshot of the QM ground state
calculations and perform a single point calculation at the S1 state to find the
next state with the highest oscillator strength.

The unnormalized probabilities are determined using
    \begin{equation}
      P'(\Omega_e) = f_{ge}(\Omega_e) \times \frac{1}{\sqrt{2\pi \sigma^2}} \exp \left[ - \frac{(\Omega_e - \Omega)^2}{2\sigma^2} \right]
    \end{equation}
where \(\Omega\) is the energy of the laser exciatation, \(\Omega_\alpha\) the energy difference from the ground state at excited state \(\alpha\), \(f_{ge}(\Omega_e)\) is oscillator strength, and \(\sigma\) the spectral broadening.  

The probability of initially populating an excited state \(\alpha\) is then
\begin{equation}
  P(\Omega_\alpha) = \frac{P'(\Omega_i)}{\sum_i P'(\Omega_i)}
\end{equation}

\section{Results}

\subsection{Intitial Excitations}

\noindent
\begin{minipage}[c]{\textwidth}
  \centering
  \includegraphics[width=5in]{../Paper2/Images/pulse_pump/spectra.png}
  \captionof{figure}{The calculated absorption spectrum from the first excited state S\(_1\). State energies are differences from the ground state.}
  \label{s1absorption}
\end{minipage}\bigskip

\subsection{State Population Relaxation}

\noindent
\begin{minipage}[c]{\textwidth}
  \centering
  \includegraphics[width=0.9\textwidth]{../Paper2/Images/populations/solvent_comparison.png}
  \captionof{figure}[Test]{Comparison of the population decays or rises of states S\(_1\), S\(_2\), and the initial state S\(_m\) between simulations with varying number of solvents included in the QM region.}
  \label{stateDecay}
\end{minipage}\bigskip

Figure ref:fig:all-populations shows the population of each state calculated as
the number of trajectories at the state's potential energy surface over the
total number of trajectories. S\(_m\) represents the initial state calculated using
the pulse pump calculations previously done. States S\(_7\) and S\(_9\) are included as
the only other "slow" states, or states that reached a population of more than
0.05. The other states were excluded from the graph. These charts show that the
addition of the NO\(_2\) oligimors dramatically speed up the state relaxation. S\(_m\)
ranged from S\(_9\) to S\(_15\) for PPV\(_3\) and S\(_11\) to S\(_21\) for PPV\(_{3}\)-NO\(_{2}\). Figure
ref:fig:s1-populations, shows the rise of the S\(_{1}\) populations over the first
500 fs after excitation. We model these rises by fitting the curves to the
function
\begin{equation}
f(t) = \frac{Ae^{t/\tau}}{A+e^{t/\tau}} - \frac{A}{1+A}
\end{equation}
where $t$ is time, $\tau$ is the relaxation, and $A$ is a constant that
normalizes such that the populations remain between 0 and 1. The results are displayed in ref:table:s1. 
We clearly see that adding a test for trivial-nonavoided crossing slows the rate
of relaxation from a time constant of 258~fs. This is to be expected since we
are now preventing transitions (mostly downward) that should not occur. The
methanol have mixed results with regards to PPV3 and seem to slightly slow the
relaxation of PPV\(_3\)-NO\(_2\). Experiments using ultrafast spectroscopy have shown that
for PPV thin films the time constant for relaxations should be around 200 fs.
However, that was on thin films and for PPV\(_3\), the energy gap !! Average S\(_1\) ->
S\(_m\) energy gap) than in the thin film (0.8eV). Previous research using the
NAESMD framework have shown a time constant of 394 fs, but this was without the
test for trivial non-avoided crossings.

\subsection{Potential Energy Relaxation}

\noindent
\begin{minipage}[c]{\textwidth}
  \centering
  \includegraphics[width=0.5\textwidth]{../Paper2/Images/potential_energies/solvent_comparison.png}
  \captionof{figure}{Potential energy difference from the intial ground state during dynamics averaged over trajectories.}
  \label{}
\end{minipage}\bigskip

\subsection{Bond Length Alternation}

\subsection{Torsional Angle Relaxation}

\noindent
\begin{minipage}[c]{\textwidth}
  \centering
  \includegraphics[width=5in]{../Paper2/Images/dihedral/solvent_comparison.png}
  \captionof{figure}{Dihedral angles of PPV\(_{3}\)-NO\(_{2}\) with varying number of solvents included in the QM region.}
  \label{dihedralNonadiabatic}
\end{minipage}\bigskip

\subsection{Widberg Bond Relaxation}

\subsection{Bond Orders}
\noindent
\begin{minipage}[c]{\textwidth}
  \centering
  \includegraphics[width=5in]{../Paper2/Images/bond_order/solvent_comparison.png}
  \captionof{figure}[Wiberg Bond Orders in Nonadiabatic Dynamics]{The Wiberg Bond Orders averaged over the ensemble of trajectories for select bonds for PPV\(_3\)-NO\(_2\) with various number of solvents included in the QM region.}
  \label{bondOrderNonadiabatic}
\end{minipage}\bigskip
% Modified from old template.

% \begin{algorithm}% Example showing the weird "algorithm" environment works...
%     \captionof{algorithm}{Test Caption}
% \end{algorithm}
% \addObject{TestStuff!}%     This is probably the command that a normal author will use to add objects.

% \chapter{EXAMPLES OF EDITOR/Author TOOLS, TABLES, AND IMAGES}% Notice that we can use chapter/section etc breaks in the master file if we want, and then use \input instead of \include to avoid unneccessary page breaks.
% \input{editorAndAuthorRemarks}%     Stuff about using editorRemark and authorRemark commands
% \input{includingTablesExamples}%    Stuff about using Tables.
% \input{includingImagesExamples}%    Stuff about using Images.

\end{document}
