\chapter{Theoretical Methods} \label{theoreticalMethods}

\section{Electronic Structure}\label{secular}

The goal of computational chemistry is to solve the Schr\(\ddot{\text{o}}\)dinger equation.
Solving it completely is only possible for very small subsets of possible situations.
In most cases, significant approximations must be made.
One of the more common such approximations is to represent the total single electron molecular orbitals contribution to the many-electron wavefunction as a linear combination of atomic orbitals (LCAO).
\begin{equation}
  \Phi=\sum_{i}c_i\phi_i
\end{equation}
where \(\Phi\) is the molecular spatial orbital, \(c_i\) the coefficient, and \(\phi_i\) the atomic orbitals.
Atomic orbitals are often designed to resemble hydrogen-like orbitals and are themselves often composed of a linear combination of Guassians to simplify integrations.
Inclusion of the spin creates the spin-orbital
\begin{equation}
  \chi = \Phi \sigma
\end{equation}
where the spin \(\sigma\) can be either \(\alpha\) or \(\beta\).

For each single electron molecular orbital, the Schodinger equation can be written as
\begin{equation} \label{eq:oneeenergy}
  E(\chi) = \frac{\left<\right.\chi\left|\right.\bm{H}\left.\right|\chi\left>\right.}{\left<\right.\chi\left.\right|\left.\chi\left.\right.\right>}
\end{equation}
where $\mathbf{H}$ is the Hamiltonian and $E$ the energy of the single electron orbital.
We can expand the numerator and denominator of the right-hand side of equation \ref{eq:oneeenergy}

\begin{align}
  \label{eq:variation1}
  \left<\right.\chi\left|\right.\bm{H}\left.\right|\chi\left>\right.&=
  \left( \sum_{i} c_i \phi_i \right) \mathbf{H} \left( \sum_j c_j \phi_j \right) &
  \left<\right.\chi\left.\right|\left.\chi\left.\right.\right>&=
  \left( \sum_{i} c_i \phi_i \right) \left( \sum_j c_j \phi_j \right)  \\
  &= \sum_{ij} c_{i}c_j H_{ij} & &= \sum_{ij} c_{i}c_j S_{ij} 
  \label{eq:variation2}
\end{align}

Taking the partial derivatives of both sides with respect to coefficient of molecular orbital a in
equation \ref{eq:variation2} provides us with

\begin{align}
  \label{eq:variationexpansion}
  \frac{\partial}{\partial c_{\alpha}}
  \left<\right.\chi\left|\right.\bm{H}\left.\right|\chi\left>\right.&=
  2c_\alpha H_{\alpha \alpha} + \sum_{\alpha j \neq \alpha} 2c_j H_{\alpha j} &
  \frac{\partial}{\partial c_{\alpha}}
  \left<\right.\chi\left.\right|\left.\chi\left.\right.\right>&=
  2 c_\alpha S_{\alpha\alpha} + \sum_{\alpha j \neq \alpha} c_j S_{\alpha j}
\end{align}

If we multiply both sides of equation \ref{eq:oneeenergy} by
$\left<\right.\chi\left.\right|\left.\chi\left.\right.\right>$ and
take the partial derivative with respect to $c_{\alpha}$,

\begin{align}
  \frac{\partial}{\partial c_{\alpha}}
  \left( E \left<\right.\chi\left.\right|\left.\chi\left.\right.\right> \right)&=
  \frac{\partial}{\partial c_{\alpha}}
  \left<\right.\chi\left|\right.\bm{H}\left.\right|\chi\left>\right. \\
  \label{eq:variation3}
  E \frac{\partial \left<\right.\chi\left.\right|\left.\chi\left.\right.\right>}{\partial c_{\alpha}}
  + \left<\right.\chi\left.\right|\left.\chi\left.\right.\right> \frac{\partial E}{\partial c_{\alpha}} &=
  \frac{\partial}{\partial c_{\alpha}}
  \left<\right.\chi\left|\right.\bm{H}\left.\right|\chi\left>\right.
\end{align}
we can minimize $E$ by rearranging equation \ref{eq:variation3}

\begin{equation}
  \frac{\partial E}{\partial c_{\alpha}} =
  \frac{1}{\left<\right.\chi\left.\right|\left.\chi\left.\right.\right>}
  \left[
    \frac{\left<\right.\chi\left|\right.\bm{H}\left.\right|\chi\left>\right.}
         {\partial c_{\alpha}}
         -E \frac{\left<\right.\chi\left.\right|\left.\chi\left.\right.\right>}
         {\partial c_{\alpha}}
         \right] = 0.
\end{equation}

Substituting our results from equation \ref{eq:variationexpansion} and
dividing by common multipliers, we find

\begin{equation}
  c_{\alpha} H_{\alpha \alpha} + \sum_{\alpha j \neq \alpha} c_j H_{\alpha j} -
  E \left( c_{\alpha} S_{\alpha \alpha} + \sum_{\alpha j \neq \alpha} c_j S_{\alpha j} \right) = 0
\end{equation}

\begin{equation}
  c_{\alpha} H_{\alpha \alpha} + \sum_{\alpha j \neq \alpha} c_j H_{\alpha j} -
  E \left( c_{\alpha} S_{\alpha \alpha} + \sum_{\alpha j \neq \alpha} c_j S_{\alpha j} \right) = 0
\end{equation}

which is often referred to as the matrix form of the Schrodinger
equation.  A more intuitive understanding of the equation may be had
if we expand out for $\alpha=1-3$.

\begin{equation} \label{eq:SchrodingerMatrix}
  \begin{bmatrix}
    H_{11}-ES_{11} & H_{12}-ES_{12} & H_{13}-ES_{13} \\
    H_{21}-ES_{21} & H_{22}-ES_{22} & H_{23}-ES_{23} \\
    H_{31}-ES_{31} & H_{32}-ES_{32} & H_{33}-ES_{33}
  \end{bmatrix}
  \begin{bmatrix}
    c_1 \\
    c_2 \\
    c_3
  \end{bmatrix} = 0
\end{equation}
This equation can be rewritten generally as
\begin{equation}
  \mathbf{H}\vec{c} = E \mathbf{S} \vec{c}.
\end{equation}
and is referred to as the secular equation.
The eigenvalues corresponding to the energies of the molecular orbitals,
whose characteristics are determined by the atomic coefficients in the
corresponding eigenvector.\cite{engel2012quantum}

\section{Hartree Fock}
To solve the secular equation we need to describe the Hamiltonian.
We begin with the generalized Hamiltonian of a molecular system,\cite{engel2012quantum}
\begin{align} \label{eq:fullhamiltonian}
  \begin{split}
    \mathbf{H} =& -\frac{\hbar^2}{2m_e}\sum_i^{electrons}\nabla_i^2-\frac{\hbar^2}{2}\sum_{A}^{nuclei}\frac{1}{M_{A}}\nabla_{A}^2 - \frac{e^2}{4\pi\varepsilon_0} \sum_i^{electrons}\sum_A^{nuclei}\frac{Z_A}{r_{iA}} \\
    & + \frac{e^2}{4\pi\varepsilon_0}\sum_{i}^{electrons}\sum_{j<i}^{electrons}\frac{1}{r_{ij}} + \frac{e^2}{4\pi\varepsilon_0}\sum_{A}^{nuclei}\sum_{B<A}^{nuclei}\frac{Z_AZ_B}{R_{AB}}
  \end{split}
\end{align}
where $A$ and $B$ are summed over all the nuclei, and the $i$ and $j$ are summed over the electrons. 
\(m_e\) and \(M_A\) are the masses of the electron and nuclei repsectively and $Z$ the charge of the nuclei.

With this Hamiltonian, the secular equation is near impossible to solve without some approximations.
The one most relevant to our work is the adiabatic approximation, also known as the Born-Oppenheimer approximation.
The Born-Oppenheimer approximation assumes electrons move so much quicker than the nuclei that we can set the second term of equation \ref{eq:fullhamiltonian} to zero and the last term to a constant. \cite{born1954dynamical,born1927quantentheorie}
We can then rewrite the electron as behaving parametrically on the coordinates of the nuclei such that our total wavefunction can be split into electronic and nuclear components
\begin{equation}
  \Psi_{total} = \sum_\alpha\psi_\alpha^{electron}(r;\mathbf{R})\psi_\alpha^{nuclei}(\mathbf{R}).
\end{equation}
The potential energy surface can be extrapolated by applying the electronic Hamiltonian $H_e$ to the wavefunction and then adding nuclear repulsion for an array of nuclear geometries.
In the mean-field approximation, each electron feels the average potential of all the other electrons, such that the second term in the electronic hamiltonian from equation \ref{eq:helectric} our total Hamiltonian becomes $\sum_i^{electrons} V_{average}(i)$.
The electronic parts the Hamiltonian are now decoupled, and we write the total Hamiltonian now as a sum of individual electron Hamiltonian's plus a nuclear-nuclear repulsion constant.
\begin{align}
  \label{eq:helectric}
  \mathbf{H}_e =& -\frac{\hbar^2}{2m_e}\sum_i^{electrons}\nabla_i^2 + \sum_i^{electrons} V_{average}(i) - \frac{e^2}{4\pi\varepsilon_0} \sum_i^{electrons}\sum_A^{nuclei}\frac{Z_A}{r_{iA}} \\
  \mathbf{H}_N =& -\frac{\hbar^2}{2}\sum_{A}^{nuclei}\frac{1}{M_{A}}\nabla_{A}^2  + \frac{e^2}{4\pi\varepsilon_0}\sum_{A}^{nuclei}\sum_{B<A}^{nuclei}\frac{Z_AZ_B}{R_{AB}}
\end{align}
We will continue this chapter in atomic units where these equations become
\begin{align}
  \label{eq:helectric}
  \mathbf{H}_e =& -\frac{1}{2}\sum_i^{electrons}\nabla_i^2 + \sum_i^{electrons} V_{average}(i) -  \sum_i^{electrons}\sum_A^{nuclei}\frac{Z_A}{r_{iA}} \\
  \mathbf{H}_N =& -\frac{\hbar^2}{2}\sum_{A}^{nuclei}\frac{1}{M_{A}}\nabla_{A}^2  + \sum_{A}^{nuclei}\sum_{B<A}^{nuclei}\frac{Z_AZ_B}{R_{AB}}
\end{align}
In actuality, the electrons of one orbit will affect electrons of the orbit of another.
The electrons will repulse each-other and their paths will change accordingly, thereby reducing the overall energy.
This approximation to the method fails to take this into account.
We call the difference between the actual energy $E$ and the Hartree-Fock energy $\epsilon$ the
coulomb correlation energy $E_{corrrelation}$.
%There have been numerous ways developed to help alleviate this problem, including perturbation theory, coupled cluster theory, and higher lever configuration interaction.

In most simulations, more than a single electron needs to be considered.
In these systems, the total electron wavefunction must satisfy the Pauli-Exclusion principle.
We should treated treat all electrons as indistinguishable, no more than one electron per set of quantum numbers, and the sign must invert for any exchange of electrons.
We can fulfill that requirement if we assume that a total electron wavefunction is a single slater-determinant of single-electron molecular orbitals
\begin{equation} \label{eq:slater-determinant} \psi(\bm{r};\bm{R}) =
  \left|p \cdots s\right> = \frac{1}{\sqrt{N!}}
  \begin{vmatrix}
    \chi_{p}(\bm{r}_1) & \cdots & \cdots \chi_{s}(\bm{r}_1) \\
    \vdots             & \ddots         &       \vdots      \\
    \chi_{p}(\bm{r}_n) & \cdots & \cdots \chi_{s}(\bm{r}_n)
  \end{vmatrix},
\end{equation}
where \(\psi\) is the total many electron wavefuntion that depend parametrically on the nuclear coordinates due to the Born-Oppenheimer approximation.
The $p \cdots s$ are the subscripts of the single electon molecular orbitals, and $1 \cdots n$ are the indices for the electrons.

Things simplify greatly if the molecular orbitals are othornormal to each other. $\left<\right.i\left|\right.j\left>\right. = \delta_{ij}$.
Intuition tells us that because of the Hamiltonian is an operator that acts on at most 2 electrons at a time, and the electron orbitals are orthonormal, any perturbation beyond 2 will integrate to 0.
In fact, there's a whole set of rules to reduce electron integral summations called the Slater-Condon rules.
\begin{enumerate}
\item
  $ \left | \cdots mn \cdots \right > \rightarrow \left | \cdots mn
  \cdots \right > \Rightarrow \sum_i \left< i \right| h \left| i
  \right> + \frac{1}{2} \sum_{ij} \left( \left< ij | ij \right> - \left< ij | ji \right> \right) $
\item
  $ \left | \cdots mn \cdots \right > \rightarrow \left | \cdots pn
  \cdots \right > \Rightarrow \left< m \right| h \left| p \right> +
  \sum_{i} \left( \left<mi | pi \right> - \left<mi | ip \right> \right) $
\item
  $ \left | \cdots mn \cdots \right > \rightarrow \left | \cdots pq
  \cdots \right > \Rightarrow \left< mn | pq \right> - \left< mn | qp \right> $
\item
  $ \left | \cdots lmn \cdots \right > \rightarrow \left | \cdots pqr
  \cdots \right > \Rightarrow 0 $,
\end{enumerate}
where the first arrow represents the perturbations of electrons. \(\left| \cdots mn \cdots \right> \rightarrow \left| \cdots pn \cdots \right>\) would present a perturbation of a single electron and
\(h\) is the core electron Hamiltonian
\begin{equation}\label{eq:CoreElectron}
  h(i) = -\frac{1}{2}\nabla_i^2 - \sum_{A=1}^N \frac{Z_A}{r_{iA}}
\end{equation}
The integral rule for the two electron integrals is
\begin{equation}
  \left< ij | kl \right> = \int dx_1 dx_2 \chi_i^*(x_1) \chi_j^*(x_2) \frac{1}{r_{12}} \chi_k(x_1) \chi_l(x_2)
\end{equation}

Using these rules and a bit of algebra, the Hamiltonian simplifies to what's called the Fock operator with elements
\begin{equation}\label{eq:Fockelement}
  F_{\mu\nu} = h_{\mu\nu}
  + \sum_{\lambda \sigma} \rho_{\lambda \sigma}
  \left(
  \left< \mu \lambda \right| \nu \sigma \left>\right.
  - \frac{1}{2} \left< \mu \lambda \right| \sigma \nu \left>\right.
  \right)
\end{equation}
where \(\rho_{\lambda \sigma}\) is the densitity matrix
\begin{equation}
  \rho_{\lambda \sigma} = \left< \psi \right| c_\lambda^\dagger c_\sigma \left | \psi \right>.
\end{equation}
We can now substituted $\mathbf{F}$ for $\mathbf{H}$ in equation \ref{eq:SchrodingerMatrix} to produce the Roothan-Hall equation
\begin{equation}
  \mathbf{Fc}=\varepsilon\mathbf{Sc},
\end{equation}
where $\varepsilon$ has replaced $E$ to be the orbital Hartree-Fock energies.
We simplify this further by using the semi-empirical AM1, which uses predetermined factors for the four -term integrations as discussed later in the semi-empirical section of this chapter.
We can now apply the variational method to determine the coefficient of the wavefunction.
First, a trial density function is chosen, which is equivalent to a trial coefficient vector.
We then solve the Roothan-Hall equation, save the lowest eigenvalue energy and use the corresponding coefficient vector to create a density function for another iteration.
We compare the energy differences between iterations until it's less than a chosen value. 
Indices i and j are summed over all electrons.

\section{Configuration Interaction}\label{CI}
	The previous calculations result in a slater determinant filled with molecular orbitals that approximates the ground state.
	We must perform some additional steps using the appropriately named post-Hartree-Fock Methods to determine the excited states.
	In this work, we use the configuration interaction methodology.

	The Hartree-Fock's slater determinant, \(\psi_0\), contains the lowest energy molecular orbitals.
	These filled orbitals are known as the occupied orbitals which we label with letters ab....
	The other available orbitals that weren't filled are considered virtual labeled ij....

	New determinants can be made by swapping virtual and occupied orbitals.
	For example
	\begin{equation}
	  \psi_c^i
	\end{equation}
	would be a determinant created by swapping the occupied orbital \(c\) with the virtual orbital \(i\) and
	\begin{equation}
	  \psi_{cd}^{ij}
	\end{equation}
	would be a determinant created by swapping occupied orbitals \(c\) and \(d\) with orbitals \(i\) and \(j\).

	For K occupied orbitals, only K swaps can be made for a single determinant.
	For each molecular orbital, there are two spin states \(\alpha\) and \(\beta\) which means for K orbitals, and N electrons, there are
	\begin{equation}
	  2K \choose N
	\end{equation}
	The full CI wavefunction, \(\Psi\), is linear combination of all of these determinants.
This method provides the exact solution to the Schr\(\ddot{\text{o}}\)dinger equation within the basis set.
	The choose function limits the use full CI to small molecules.

	For larger molecules, we only include the ground state determinant and either the singles (configuration interaction singles (CIS)), the doubles (CID), or both (CISD).
	For CIS, the new wavefunction can be written as

	\begin{equation}
	\Psi_{CIS} = c_0\psi_0 + c_a^i\sum_i^N\sum_a^{K-N}\psi_a^i
	\end{equation}
	where \(c_0\) and \(\psi_0\) are the coefficients and determinant for the Hartree-Fock ground state respectively.

	To solve for the coefficients, we use a similar method of solving an eigenvalue equation like that performed in \ref{secular}.
	\begin{equation}\label{eq:CIS}
	  \bm{H}\vec{c} = \bm{e} \bm{S} \vec{c}
	\end{equation}
	where
	\begin{align}
	  H_{ji} &= \left<\psi_b^j \right| \bm{H} \left| \psi_a^i \right> \label{eq:CISMatrix}\\
	  S_{ji} &= \left<\psi_b^j | \psi_a^i \right>
	\end{align}
	are the Hamiltonian \(\bm{H}\) and overlap \(\bm{S}\) matrices.
	When diaganolized, \(\vec{c}\) and \(\bm{e}\) are the coefficients and the energies of the CIS wave functions composed as a linear sum of the exchange determinants.

When using CIS, the addition of the single exchange determinants does not affect the ground state
while the linear combination of the mixed singly excited determinants accounts for some of electron correlation in the excited states.

\section{Semiempirical Methods}
Solving the equations for the Hartree Fock method and Configuration Interaction requires the integrations of many two-electron integrals.
Using hydrogen-like slater orbitals for these integrations becomes infeasible.
It is common to approximate these orbitals using Gaussian functions.
Each atomic orbital is a linear combination of Guassians, and each molecular orbital is using a slater determinant of these combinations.
However, computational costs still limit the solving of the Shr\(\ddot{o}\)dinger equations in this basis to but a couple of atoms.
For larger systems, we require further approximations.
A standard approximation is to replace the overlap matrix S with the unit matrix in the Roothan hall equation
\begin{equation}
\mathbf{F} \vec{c} = \bm{\epsilon}\mathbf{S}\vec{c}
\end{equation}
and only treat the valence electrons quantum mechanically. \cite{christensen2016semiempirical}
The approach is called the zero-differential overlap approximation.
This action reduces the cost order of the integrations from \(O(N_{\text{electrons}})^4\) to \(O(N_{\text{valence electrons}})^2\).
We can further reduce the computational costs by parameterizing the remaining \((ii|jj)\) integrals to experimental data as done in the complete negelect of differential overlap methods.
A typical correction is to reintroduce parameterized integral approximations for \((ij|kl)\) where ij are electrons on one atom, and kl another. \cite{pople1965approximate}
The neglect of diatomic differential overlap approximation further corrects by replacing the core-core interactions with Z\(_A\) Z\(_B\) (core\(_a\) core\(_a\) | core\(_b\) core\(_b\)).
The neglect of diatomic differential overlap is the foundation for most of the semiempirical methods.

In this work we use a modified version of the neglect of diatomic differential overlap approximation known as the Austim Model 1 (AM1) Hamiltonian. \cite{Dewar1985}
In this approximation, the integrals of type \((ij | kl )\) are approximated using the multipole moments. \cite{Dewar1985}
The core-core interactions are modified to
\begin{align}
\begin{split}
E_{core-core}^{AM1} = &E_{core-core}^{MNDO} \frac{Z_{A} Z_{B}}{R_{AB}} [\\
  &\sum_i (K_{A_i}, \exp(L_{A_i}, (R_{AB} - M_A)^2)) \\
+ &\sum_i (K_{B_i}, \exp(L_{B_i}, (R_{AB} - M_B)^2))]
\end{split}.
\end{align}\cite{christensen2016semiempirical}

AM1 has been used successfully for organic conjugated polymers such as the one we analyze in chapters 4 and 5. \cite{ozaki2019molecular,silva2010benchmark,moran2003excited,cornil1994optical,weingart2012modelling}
NEXMD can also utilize time dependent density functional theory (TDFT),
but we restrict our use to AM1 CIS as is commonly applied in the use of the NEXMD software package on organic conjugated polymers. \cite{tretiak2003resonant}

\section{QM/MM}
We use SANDER's QM/MM implementation to provide approximations of the solvent interactions.\cite{Walker2008}
SANDER's combined QM/MM Hamiltonian represents MM atoms as point charges and QM atoms as electronic wave-functions.
The effective Hamiltonian uses the aforementioned hybrid approach
\begin{equation}
  \mathbf{H}_{eff} = \mathbf{H}_{QM} + \mathbf{H}_{MM} + \mathbf{H}_{QM/MM}
\end{equation}
where \(\mathbf{H}_{QM}\), \(\mathbf{H}_{MM}\), \(\mathbf{H}_{QM/MM}\) are the Hamiltonians for the QM to QM, MM to MM, and QM to MM hybrid interactions.
We do not consider \(\mathbf{H}_{MM}\) during the electronic calculations due to its independence from the electronic distribution.
\(\mathbf{H}_{QM}\) is the electronic Hamiltonian used in vacuum QM SCF calculations.
\(\mathbf{H}_{QM/MM}\) represents the interactions between the QM charge density and MM atoms treated as point charges.
For computational efficiency we limit this interaction by a distance cuttoff, set by the user, generally in the range of 10-16 \(\AA\) from the perimeter QM atoms.
For short-range interactions, we expand the hybrid \(\mathbf{H}_{QM/MM}\) into
\begin{align}\label{eq:qmmm}
  \begin{split}
    \mathbf{H}_{QM/MM} = &- \sum_i \sum_m q_m \hat{h}_{electron} (\vec{r}_i,  \vec{r}_m)\\
    &+ \sum_q \sum_m q_q q_m \hat{h}_{core} (\vec{r}_q, \vec{r}_m)\\
    &+ \sum_m \sum_q \left( \frac{A_{qm}}{r_{qm}^{12}} - \frac{B_{qm}}{r_{qm}^6} \right)
  \end{split},
\end{align}
where \(i\) is the electron, \(m\) the MM atom, and \(q\) the combined nuclei and core electrons of the QM atoms.
A and B are the Lennard-Jones interaction parameters where \(r_{qm}\) is the distance between the MM and QM atoms.
\(q\) is the charge, and \(r\) is the coordinate vector.
\(\hat{h}_{core}\) represents the electronic interactions between the MM charges and the core of the QM atoms.
\(\hat{h}_{electron}\) represents the interactions between the MM charges and either the charge density of the QM region when using semi-emprical methods or using the Mulliken charges in the case of DFT.

The short-range interactions, shown as the second term in equation \ref{eq:qmmm} can be straightfowardly added to the Fock matrix from equation \ref{eq:Fockelement}
\begin{equation}
  F_{\mu\nu}^{SRC} = F_{\mu\nu} + \sum_{m} \frac{Z_{m}}{r_{\nu m}}
\end{equation}
where \(F_{\mu\nu}^{SRC}\) are the elements of the short range electrostatic corrected Fock matrix, \(\mu\) and \(\nu\) the electronic indices, and \(m\) the nulclear indices for the classical atoms.

Long-range interaction, from those outside the cutoff, considered vital for the understanding of solvent effects, are treated using SQM’s implementation of Particle Mesh Ewald.\cite{darden1993particle}
Trajectories use periodic boundary conditions to simulate an explicit solution, treating the system box as cells repeated infinitely many times in all directions.
Particle Mesh Ewald calculations then determine the long-distance interactions of these periodic boxes, treating the charges and potentials in the long-range inter-box distances as sums in Fourier space treating atoms in the QM region of these calculations as Mulliken point charges.\cite{essman1995smooth}
Once the sums are complete, SQM performs a fast Fourier transformation to obtain the long-range corrections to the energies.
These corrections can be added to the short-range corrected Fock matrix to get the complete QM/MM corrected Fock matrix
\begin{equation}
  F_{\mu\nu}^{QMMM} = F_{\mu\nu}^{SRC} + \frac{\partial}{\partial_{\rho\nu}}\left(\Delta E^{PBC}[Q,Q] + \Delta E^{PBC}[Q,q] \right)
\end{equation}
where \(E^{PBC}[Q,Q]\) describes the periodic energies from the QM atoms treated as Mulliken charges and \(\Delta^{PBC} [Q,q]\) the periodic energies from the MM atoms.
\(\Delta E^{PBC}[Q,Q]\) depends on the Mulliken charges of the QM atoms which are dependent on the trace of the density matrix which isn't known until the Roothan-Hall equation is solved, therefore \(\Delta E^{PBC}[Q,Q]\) is solved for at every step of the SCF procedure.
The correction from \(\Delta^{PBC} [Q,q]\) is simply the potential from the periodic MM atoms and is not dependent on the Mulliken charges of the QM atoms and as such can be added to the Fock matrix before the SCF routine along with the short-range electrostatic correction.

\section{Adiabatic Dynamics}
    For excited-state calculations, we implement the Collective Electronic Oscillator (CEO) approach developed by Mukamel and coworkers, which solves the adiabatic equation of motion of a single electron density matrix.\cite{tretiak02_densit_matrix_analy_simul_elect,tommasini2001electronic}
    We define the single-electron density matrix
    \begin{equation}
      (\rho_{g\alpha})_{nm}(t) = \left< \psi_\alpha (t) \right| c_m^\dagger c_n \left | \psi_g (t) \right>
    \end{equation}
    where \(\psi_g\) and \(\psi_\alpha\) are the single-electron wave functions of the ground-state and \(\alpha\) state respectively.
    \(c_m^\dagger (c_n)\) is the creation(annihilation) operator summed over the atomic orbital \(m\) and \(n\), whose size is determined by the basis set.
    The basis set coefficients of these atomic orbits are calculated in the previous SCF step and account for the presence of solvents.
    The CIS approximation is applied, creating the normalization condition 

    \begin{equation}
      \sum_{n,m} (\rho_{g\alpha})^2_{n,m} = 1
    \end{equation}

    Recognizing that \(\rho_{g\alpha}\) represents the transition density from the ground to the \(\alpha\) state, we solve the Liouville equation of motion 

    \begin{equation}\label{eq:liouville}
      \mathcal{L}\bm{\rho}_{0\alpha} = \Omega \bm{\rho}_{0\alpha},
    \end{equation}
    with \(\mathcal{L}\) being the two-particle Liouville operator and \(\Omega\) the energy difference between the \(\alpha\) state and the ground state.

    Equation \ref{eq:liouville} can be shown to be a genarallization of the CIS method shown in section \ref{CI}.
    In a molecular orbital representation, equation \ref{eq:liouville} becomes the first-order random phase approximation (RPA)
    \begin{equation}
      \begin{bmatrix} 
        \mathbf{A} & \mathbf{B} \\
        -\mathbf{B} & -\mathbf{A}
      \end{bmatrix}
      \begin{bmatrix} 
        \mathbf{X}\\
        -\mathbf{Y}
      \end{bmatrix} = \Omega
      \begin{bmatrix} 
        \mathbf{X}\\
        -\mathbf{Y}
      \end{bmatrix}
    \end{equation}
    where the transition density matrix, \(\bm{\rho}_{0\alpha}\), has been split into its particle-hole, (\(\mathbf{X}\)), and hole-particle, (\(\mathbf{Y}\)), components.
    \(\mathbf{A}\) is identical to the CIS matrix in equation \ref{eq:CISMatrix}.
    \(\mathbf{B}\) represent the higher order terms.
    Dropping the higher terms provides the Tamm-Dancoff approximation \cite{dunning1967nonempirical}
    \begin{equation}
      \mathbf{A} \mathbf{X} = \Omega \mathbf{X}.
    \end{equation}
    which is the same as the CIS equation \ref{eq:CIS} after the negelect of differencial overlap approximation is applied.

    We can avoid the full diaganolization of equation \ref{eq:liouville} because the Liouville operator can be found analytically using
    \begin{equation}
      \mathcal{L} \bm{\rho}_{o\alpha} = \left[ \vec{\nabla} \mathbf{F}(\bm{\rho}_{00}),\bm{\rho}_{0\alpha} \right] +
      \left[ \vec{\nabla} \mathbf{V}(\bm{\rho}_{0\alpha}), \bm{\rho}_{00} \right]
    \end{equation}

    where \(\mathbf{F}\) is the Fock operator and \(\mathbf{V}\) is the column interchange operator or the second term from equation \ref{eq:Fockelement}
    \begin{equation}\label{eq:ColumnInterchange}
      \mathbf{V}(\bm{\rho}_{0\alpha}) = \sum_{\lambda \sigma} P^{0\alpha}_{\lambda \sigma}
      \left(
      \left< \mu \lambda \right| \nu \sigma \left>\right.
      - \frac{1}{2} \left< \mu \lambda \right| \sigma \nu \left>\right.
      \right)
    \end{equation}
    where \(P^{0\alpha}_{\lambda \sigma}\) are the elements of the transition density matrix between the ground state and excited state \(\alpha\).
We use the Davidson technique to diagonlize these Liouville equation of motions to reduce the computational costs from an otherwise O(6) to O(3). \cite{nelson2011nonadiabatic}

    The QM-QM interaction forces are then calculated using the gradient of the ground state and excited state energy QM energies. 
    \begin{equation} \label{eq:NEXMDForces}
      \vec{\nabla} E_\alpha = \vec{\nabla} E_0 + \vec{\nabla}\Omega_\alpha
    \end{equation}
    These gradients are calculated analytically, allowing a significant efficiency advantage over other numerical methods.
    The gradient of the ground state being can be shown to be
    \begin{equation}
      \vec{\nabla}E_0 = \frac{1}{2} \text{Tr} \left[ \left(\vec{\nabla} \mathbf{h} + \vec{\nabla} \mathbf{F}(\bm{\rho}_{00}) \right)\bm{\rho}_{00} \right]
    \end{equation}
    where \(\mathbf{h}\) is the core electron hamiltonian from equation \ref{eq:CoreElectron}.
    The gradient of the excited state transition energies can be described by
    \begin{equation}
      \vec{\nabla}\Omega_\alpha = \text{Tr} \left( \vec{\nabla}\mathbf{F}(\bm{\rho}_{00}) \left( \bm{\rho}_{\alpha\alpha} - \bm{\rho}_{00} \right) \right) + \text{Tr} \left( \vec{\nabla}\mathbf{V} (\bm{\rho}_{0\alpha}^\dagger) \bm{\rho}_{0\alpha} \right).
    \end{equation}
We then add the QM-MM interaction forces using the density matrix from current state \(\alpha\) to return the final QM forces. The reciprical of these forces are added to the MM atoms which otherwise derive their forces from classical force fields.

\section{Non-Adiabatic Dynamics}

The MDQT approach utilized in this work is as a modified version of the Tully Surface Hopping method.\cite{tully2012perspective, tully1990molecular,Tully1998}
Here the quantum wave function is approximated using a swarm of independent trajectories.
These trajectories propagate along adiabatic surfaces;
However, between time steps, these trajectories are allowed to transition from one state to another in a Monte Carlo-like fashion.
That number of trajectories in any given state corresponds to that state's quantum probability.

We define the Hamiltonian

          \begin{equation} \label{eq:tullyHamiltonian} \mathbf{H} = \mathbf{T}(\mathbf{R}) +
            \mathbf{H}_{el}(\mathbf{r},\mathbf{R})
          \end{equation}
          where \(\mathbf{T}(\mathbf{R}) \) is the nuclear kinetic energy operator and \(\textbf{H}_{el}\) is the electronic Hamiltonian.

          We then expand the the total wavefunction, \(\Psi\) into orthonormal adiabatic state wavefunctions \(\psi\)
          \begin{equation}
            \Psi(\textbf{r}, \textbf{R}, t) = \sum_j c_j(t)\psi_j(\textbf{r}; \textbf{R}) = c_j \left| \psi_j \right>
          \end{equation}
          where \(\textbf{r}\) and \(\textbf{R}\) are the electronic and nuclear coordinates respectively.
          \(c_j\) are complex expansion coefficients.
          Substitution into the Shr\(\ddot{o}\)dinger equation yeilds

          \begin{align}
            i\hbar \frac{\partial}{\partial t} c_j \left | \psi _j \right> &= \mathbf{H} c_j \left | \psi_j \right>\\
            i\hbar \dot{c}_j \left | \psi \right> + i\hbar c_j \left| \dot{\psi}_j \right> &= \mathbf{H} c_j \left| \psi_j \right>\\
          \end{align}
          where we can now apply it another state \(\psi_i\) on the left.
          \begin{align} \label{eq:dcoefficient}
            i\hbar \dot{c_j} \left< \psi_i | \psi_j \right> + i\hbar c_j \left< \psi_i | \dot{\psi}_j \right> &= c_j \left< \psi_i | \mathbf{H} | \psi_j \right>\\
            \sum_j i\hbar \dot{c_j} \left< \psi_i | \psi_j \right> &= \sum_j \left(c_j \left< \psi_i | \mathbf{H} | \psi_j \right> - i\hbar c_j \left< \psi_i | \dot{\psi}_j \right> \right)\\
            i\hbar \dot{c_i} &= \sum_j \left(c_j \left< \psi_i | \mathbf{H} | \psi_j \right> - i\hbar c_j \left< \psi_i | \dot{\psi}_j \right> \right)
          \end{align}
          where we now made the sum explicit.
          The second term on the right \(\left< \psi_i | \dot{\psi}_j \right>\) is referred to as the non-adiabatic adiabatic coupling and represents the coupling between states i and j.
          This can be rewritten as 
          \begin{equation} \label{eq:tullyS3}
            \left<\psi_i\right|\dot{\psi}_j\left.\right>=\left<\psi_i\right|\frac{d\mathbf{R}}{dt}\frac{d}{d\mathbf{R}}\left|\psi_j\right>=\dot{\mathbf{R}}\cdot\mathbf{d}_{ij}(\mathbf{R})
          \end{equation}
          effectively separating the coupling term into the nuclear velocity vector \(\dot{R}\) and another vector referred to as the non-adiabatic coupling vector 
          \begin{equation} \label{eq:tullynacoupling} 
            \mathbf{d}_{ij}\mathbf(R) =
            \left<\psi_{i}(\mathbf{r};\mathbf{R})\right|\mathbf{\nabla}_{\mathbf{R}}\left.\psi_j(\mathbf{r};\mathbf{R})\right>.
          \end{equation}
          Equation \ref{eq:tullyS3} clearly shows that the coupling is strongest when the non-adiabatic vector is aligned with the nuclear velocities.
          Also the coupling is proportional to the magnitude of these velocities.
          Through use of the Helmann-Feynman theorem, we can calculate the coupliing vector analytically allowing us to calculate it "on the fly". \cite{chernyak2000density,tommasini2001electronic,tretiak1996collective,tretiak2009representation,Tretiak1996,Tretiak1999}
          \begin{equation}\label{NACouplingAnalytic}
            \mathbf{d}_{ij} = \frac{\text{Tr}(\vec{\nabla}\mathbf{F}(\bm{\rho}_{00})\bm{\rho}_{ij})}
                   {\Omega_i - \Omega_j}.
          \end{equation}
     A similar method is performed to show that the non-adiabatic coupling term
\begin{equation}\label{eq:NACouplingTerm}
\dot{\mathbf{R}} \cdot \mathbf{d}_{ij} = \frac{\text{Tr}(\mathbf{F}^t(\bm{\rho}_{00}) \bm{\rho}_{ij})}{\Omega_i - \Omega_j},
\end{equation}
where the superscript \(t\) denotes the derivative in respect to time.
          This formulation clearly demonstrates that the magnitude of non-adiabatic coupling is inversely proportional to the change in energies between the states.

          To simplify notation we will let
          \begin{equation}
            \mathbf{V}_{ij} = \left< \psi_i | \mathbf{H} | \psi_j \right>
          \end{equation}

          Substituting this new notation into \ref{eq:dcoefficient} gives
          \begin{equation}
            i\hbar \dot{c_i} = \sum_j c_j \left(\mathbf{V}_{ij} - i\hbar \dot{\mathbf{R}}\cdot\mathbf{d}_{ij}(\mathbf{R}) \right)
          \end{equation}
          which can be written in terms of a state density matrix
          \begin{align}
            i\hbar a_{kl} &= c_k c_l^*\\
            i\hbar \dot{a}_{kl} &= \dot{c}_k c_l^* + c_k \dot{c}_l^* \\
            i\hbar \dot{a}_{kl} &= \sum_j \left[ a_{jl} (\mathbf{V}_{kj} - i\hbar \dot{\mathbf{R}} \cdot \mathbf{d}_{kj})
              - a_{kj} ( \mathbf{V}_{lj} + i\hbar \dot{\mathbf{R}} \cdot \mathbf{d}_{lj}^*) \right] \\
            i\hbar \dot{a}_{kl} &= \sum_j \left[ a_{jl} (\mathbf{V}_{kj} - i\hbar \dot{\mathbf{R}} \cdot \mathbf{d}_{kj})
              - a_{kj} ( \mathbf{V}_{lj} - i\hbar \dot{\mathbf{R}} \cdot \mathbf{d}_{jl}) \right]
          \end{align}
          where \(d_{lj}^* = -d_{jl}\) can be deduced from equation \ref{eq:tullynacoupling}.

          The diagonals of \(\dot{a}_{kl}\) represents the rates at which the populations of electonic states are changing
          \begin{align}
            \dot{a}_{kk} &= -\frac{i}{\hbar}\sum_j \left[ a_{jk} (\mathbf{V}_{kj} - i\hbar \dot{\mathbf{R}} \cdot \mathbf{d}_{kj})
              - a_{kj} ( \mathbf{V}_{kj} - i\hbar \dot{\mathbf{R}} \cdot \mathbf{d}_{jk}) \right].
          \end{align}
          Its worth looking further into these included terms,
          \begin{equation}
            b_{kj} = - \frac{i}{\hbar} \left(a_{jk} (\mathbf{V}_{kj} - i\hbar \dot{\mathbf{R}} \cdot \mathbf{d}_{kj}) - a_{kj} ( \mathbf{V}_{kj} - i\hbar \dot{\mathbf{R}} \cdot \mathbf{d}_{jk})\right).
          \end{equation}
          \(b_{kj}\) represents the net population flow from state \(j\) to state \(k\). If \(j = k \), \(b_{kj} = 0\), suggesting that there is no self flow.
          With a little algebra, we can show that while the state density matrices are complex, the net population flows are real
          \begin{equation} \label{eq:tullyb2a} 
            b_{kj} =
            \frac{2}{\hbar}\Im\left(a_{kj}^*\mathbf{V}_{kj}\right) - 2\Re\left(a_{kj}^*
            \dot{\mathbf{R}} \cdot \mathbf{d}_{kj}\right).
          \end{equation}
          During dynamics, between time-steps, the system can only travel along adiabatic PES. 
          These flows must thus be converted to a probability of a hop.
          The probability of a hop from state j to k can be described as
          \begin{equation} \label{eq:HopProbability}
            \text{P}_{j \rightarrow k} = \frac{\text{Population from j to k}}{\text{Original population of j}} = \frac{b_{kj} \Delta t}{a_{jj}}.
          \end{equation}
          As can be seen from equations \ref{eq:HopProbability} and \ref{eq:tullyb2a}, the chance to hop is linearly dependent on the non-adiabatic coupling term which we found earlier to increase with smaller energy differences or larger nuclear velocities. 
          These findings are similar to that found by Zener in the 1930s where the probability of a hop between two diabatic states was given by
          \begin{align}
            P_{j \rightarrow k} &= e ^{-2\pi \Gamma}\\
            \Gamma &= \frac{a_{kj}}{\hbar \frac{dq}{dt}\frac{\partial}{\partial q} (\Omega_k - \Omega_j)}
          \end{align}
     implying that the chance to hop is greater with a larger velocity in a direction that maximized the rate of reduction between the energy levels.\cite{zener1932non}

At each step we perform a montecarlo-like decision based on these probabilities in equation \ref{eq:HopProbability}.
We choose a uniform random number \(\zeta\) from 0 to 1.
A hop from j to k will occur if 

\begin{equation} \label{eq:tullyjump2} 
\sum_{l=1}^{k-1}P_{j \rightarrow l} < \zeta  \le \sum_{l=1}^{k}P_{j \rightarrow l}.
\end{equation}

Inconsistencies arise from solely using the Tully surface hopping approach.
 For example, there will in an inconsistency between the states populations and the quantum coefficients.
 When trajectories hop between various adiabatic potential energy surfaces instantaneously based off the QM state coefficients determined using the integral of the TDSE on multiple trajectories, each trajectory if unmodified will keep in phase even after spatial separation which is a non-physical occurence. \cite{joos2013decoherence,landry2011communication,nelson2013nonadiabatic}
Properly accounting for these coherences is necessary for controling charge separtion in light-harvesting devices.\cite{rozzi2013quantum}
Furthermore, if dealing with a system with a dense electronic state structure, its likely that the ordering of noncoupled states will switch during general dynamics.
Fortunately, many improvements to the surface model method has been developed since its first conception.\cite{fang1999improvement}
We apply a dechohence correction as well as a trivial crossing accounting system as performed in previous research.\cite{nelson2013nonadiabatic,fernandez2012identification}
