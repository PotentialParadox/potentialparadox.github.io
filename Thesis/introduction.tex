\chapter{Introduction} \label{introduction}

\section{Prologue}

The effects of light on the physical properties of materials has maintined the interest of mankind for as long as history itself.
Humans most likely noticed power of the sun to turn their skin red and itchy far before they even developed language.
Records show the interest of reducing the bleaching dyes.
The documents describing the mirror of Archimedes demonstrates that humans desire to harness this power dates back at leas multiple millinea.
Our understanding began to formalized in the late 1700s when Priestly experiments shined light on the processes of oxidations and sparked a curiosity that led to further experimentations with photosynthesis and photochemistry in general.
Since then, researchers have further advanced our knowledge of these effects and our ability to harness the power of light.

The ability to model these photo-energetic non-adiabatic dynamics has recently become more feasable.
We have used this ability to continue our long pursuit to understand organic photosythesis. \cite{zheng2017photoinduced,caycedo2010light}
The search for how to efficiently create and utilize sythetic organic phtosythesizer has also been a focus of interest. \cite{balzani2008photochemical,engel2007evidence}
Studies with non-adiabatic dynamics have been used to study possible light harvesting technologies. \cite{ishida11_effic_excit_energ_trans_react,katan2005effects}

A similar process can also illuminate our understanding and production of efficient custom light emitting diodes. (cite: park 1-2)
This type of modeling can also help with understanding photo-detection.
Recent works have helped understand how the the protein rhodopsin behaves in the human eye.\cite{weingart2012modelling}
Continued research can help develop more sensitive or enery efficient optical sensors. 
The modeling of theses types of dynamics currently boasts a broad academic and industrial interest. \cite{tavernelli2010nonadiabatic,tavernelli2015nonadiabatic,nelson2020non}
Ultra fast proton transfer on the time order of femtoseconds have sparked much interest in last few decades.\cite{schwartz1992direct}

\section{Qualitative Overview of Non-Adiabatic Dynamics}

\subsection{Energy Transfer}

\noindent
	  \begin{multiFigure} 
	    \addFigure{0.45}{../Oral/Images/photoexcitation.png}
	    \addFigure{0.45}{../Oral/Images/pes_chart_zoomed.png}
	    \captionof{figure}{Diagrams describing the behavior of a molecule throughout an photo-excitation event.}
	    \label{fig:jablonski}
	  \end{multiFigure}
\bigskip

Figure \ref{fig:jablonski} A shows whats referred to as a Jablonski diagrams.
S\(_0-2\) represent the potential energy surfaces for the three lowest singlet states.
T\(_1\) represents the first excited triplet state.
No vibrational or rotational modes are shown since we will treat these classically.
Immediately after an electron photon absorption, the molecule is promoted to an excited state, as can be seen by the purple arrow.
This excited state could either the one immediately above it, or it could be one the many above that one.
The decision of which state to excited to is determined by the energy of the excitation and oscillator strength.

Once the molecule is at this excited states, ignoring high temperature, it will relax back towards the ground state.
There are two primary mechanisms through which this can occur.
The first is by releasing the energy thermally either throughout the rest of the molecule or to the environment. This method is referred to as internal conversion and manifests as reductions to the vibrational and rotational modes.
The second is through photon-emission.
A photo-emission process from the first excited state to the ground state is referred to as fluorescence, and can be seen by the green arrow in the figure.
Fluorescence occur over period of nanoseconds.
Tranistion processeses from singlet states to the triplet states are possible dependent on the strength of the spin-orbit coupling, in a process called intersystem conversion.
Photo-emmission from the triplet state to the ground state would be called phosphorescence.
Phosphorescence is relatively very rare compared to fluorescence with time order ~1s.
For this reason we do not consider this behavior in our simulations.

Kashas rule states that photon-emission occurs only in appreciable yields from the lowest excited state to the ground state.\cite{Kasha1950}
This rule suggests that in most cases where an electron is excited to a state beyond the first excited state, that electron will have to relax to the first excited state by means of internal conversion.\cite{shenai2016internal}

Figure \ref{fig:jablonski} B, is a zoomed in picture of the portion of Figure 1b surrounded by the orange circle.
When the molecule is excited to S2 through photo-excitation, it will begin to relax along S2's potential energy surface following the orange arrow.
In reality, this process would be quantized and occur as a gradual reduction in the vibration and rotational modes.
In our simulations though, we treat these reduction classically and the molecule can move smoothly along the potential energy surface of each excited state. 
However, eventually the molecule traversing the potential energy surface of S2 will cross the potential energy surface of S1.
At these crossings, there is generally strong couplings between the two states.
This coupling allows the molecule to transition from S2 to S1.
A choice now needs to made whether to stay on potential energy surface of the S2 or switch to S1.

In computation chemistry it is common to assume that electrons move significantly faster than nuclei and treat the nuclei as parameters to the equations used to solve for electronic behaviors.
This assumption is known as the Born-Oppenheimer approximation and forces the molecule to traverse along a single potential enery surface making it impossible for trasitions from one excited state to another to occur.
Simulations of traversals restricted to a single potential energy surface is referred to as adiabatic dynamics.
Simulations that allow such crossings are non-adiabatic.

During ultra-fast photovolatic processes, the Born-Openheimer appoximation breaks, and accounting for non-adiabatic behavior become necessary.
These situations occur frequently within processes of interest to photochemistry and photophysics.
For example, the excitation to a non-equlibrium state followed by a relaxation through internal conversion is a process common to processes such as photosynthesis, solar-cell photo-absoprtion, optical detectors, and the excitation of the visual nerve.
Photon absorption is also a requirement in certain reactions that need that last little kick.\cite{vincent2016little}
The S\(_1\) and S\(_2\) lines figure \ref{fig:jablonski} represent crossing between potential energy surfaces.

\noindent
       \begin{multiFigure} 
	 \addFigure{0.45}{Images/probabilities.png}
	 \addFigure{0.45}{Images/ehrenfestVsTully.png}
	 \captionof{figure}[Surface Hopping vs Mean-Field]{A visual description describing the difference between surface hopping and mean-field. A) The probabilities states S1 and S2. B) The potential energies of trajectories over time. Dashed lines represent represent the potential engergies of S1, S2, and the probability weighted average during the Ehrenfest trajectory. Solid lines represent two sepearte surface hopping trajectories.}
	 \label{fig:surfaceHoppingVsMeanField}
       \end{multiFigure}
\bigskip

The two most common methods to extend the Born Oppenheimer appoximations are through a mean field, ofter referred to as Ehrenfest, or through molecular dynamics with quantum tranistions (MDQT).\cite{Hammes-Schiffer1994}
Alternative methods using mixed quantum-classical dynamics do exist and are used in the field. \cite{habershon2013ring,kapral2006progress}
In Ehrenfest methods, the forces acting on the molecule at any timestep is the population weighted average of the forces acting at all relevant excited states.
In MDQT methods only the forces of one state is used for any single time-step. \cite{prezhdo1997evaluation}
Between timesteps, the molecule is allowed to transition between states.
To simulate state populations, MDQT methods employ a swarm of independent trajectories.
Each trajectory is given a different random seed and allowed to hop between states based on the non-adiabatic couplings.
Study of the system's behavior is then done based on the statistics of the ensemble.

Figures \ref{fig:surfaceHoppingVsMeanField} A and B attempt to show the practical differences between these two methods.
The population chart on the left shows the probability of being in states S1 and S2 at some arbitrary time.
These probabilities merge to around 0.5 each at around the halfway point.

The right figure presents arbitrary state energies over the same time frame for this trajectory.
The dashed lines represent the energies along the Ehrenfest trajectory.
Blue and red represent the S2 and S1 energies repectively.
The black dashed line represents the Ehrenfest mean-field energy determined as the population weighted average energies of S1 and S1.
As the probability of state S2 drops from one, the mean-field energy diverges from that of S2.
Eventually the mean field energy becomes the average of a S1 and S2.

The solid lines represent the the energies along two separate surface hopping trajectories.
At around the halfway point, the trajectory SH-S1 hops from the S2 to S1.
Trajectory SH-S2 remains on S2.
Because these trajectories are allowed to be moved by forces generated at their respective potential energy surfaces, their energies will in general be lower than their mean field counterparts.
Notice that the average energy of the hop trajectories will also diverge from the Ehrenfest method.

\noindent
\begin{minipage}[c]{\textwidth}
  \centering
  \includegraphics[width=\textwidth]{./Images/naCrossings.png}
  \captionof{figure}[Regions of Non-Adiabatic Couplings]{Periods of trajectories where there is in general weak and strong state couplings between states S1 and S2 and well as region where the energies of S1 and S2 cross.}
  \label{fig:naCrossings}
\end{minipage}\bigskip

In this work, we model the interstate transitions using the MDQT algorithm, Tully's Fewest Switches.
The probability of hopping from one state to another is proportional to the coupling between the states known as the nonadiabatic coupling.
These nonadiabatic coupling are dependent in part on the energy differences between the states, and the nuclear velocities.
Figure \ref{fig:naCrossings} shows three approaches from potential surfaces S1 and S2.
Assume that the molecule is originally on state S2.
When the the energy differences are relatively large, with a shallow approach as in the left figure, the coupling is weak, and hops become unlikely.
When the aproach is steep, and the energy difference small, the nuclear velocities no longer become negligible, the Born Oppenheimer approximation breaks, strong coupling exists, and a respective hop become likely.
In the far right figure, the energies of the two states cross.
In general states in molecular dynamics programs are refered to based on their energy orderings.
In this situation, the orderings of these potential energy sufaces swicth and S1\(\rightarrow\)S2 and vice versa.
If no adiabatic hopping occurs, the molecule remains on the same potential energy suface.
However, the energy levels will have have switched and we need to ensure that molecule traverses along the new S1 state.
This can be done by comparing overlaps between the states between timesteps.

\subsection{Solvent Effects}
The determination of which state to excited to is strongly affected by the transition dipole moments.
These transition dipole moments are sensitive to polarization from external electronic fields or charges
The energy differences between the excited states can also be affected by these external charges due to (de)-stabalization of the dipoles.
Systems with strong electric fields occur frequently in biological systems. (cite: furukawahino 41-44)
These electric field can have profound effect on the steady state fluorescnce and absorption spectra. \cite{park2013}
The solvents in these systems can extend or shield these effects.
Solvents provide a large source of external charges that can significantly affect the non-adiabatic behavior and characteristics of a molecule.\cite{furukawa2015external}

Multiple methods to have been proposed and used to simulate these non-adiabatic processes.
These methods include treating the nuclear coordinates quantum mechanically or simiclassically, or by using a hybrid quantum mechanical, classical treatment to account for the non-adiabaticity.
One of the more popular version of the latter, and the one which we use in this work, is Molecular Dynamics with Quantum Transitions (MDQT), were the system propogates classically along adiabatic potential energy surfaces, but a quantum evalutation is made at each time step to determine whether to transition to another state.

For many areas in which nonadiabatic dynamics simulations would be of interest, solvents play a crucial role.
\cite{bagchi1989dynamics,woo2005solvent}
    In situations where ultrafast electronic relaxations occur, the electronic decay is often faster than the time for the solvent to equilibrate.
    As such, Implicit solvents, which adjusts instantaneously to any changes, become imprecise approximation.
    However performing non-adiabatic dynamics on such large systems is far too computationally expensive.
    To ease the computational cost we can employ QM/MM methodologies to perform the non-adiabatic calculation only on the areas of interest.
    Similar methods have been employed in the study of retinal photochemistry and organic semiconductors.\cite{weingart2012modelling,demoulin2017fine,heck2015multi,bayliss1954solvent}%\cite{demoulin2017fine, weingart2012modelling, heck2015multi}
    In this work we implement a new method of performing non-adiabatic QM/MM using the SANDER package AMBERTOOLS combined with the high performance Non-Adiabatic simulator NEXMD.
    We further analyze the effects of including near solvent molecules within the QM region.

\subsection{QM/MM}
	\begin{multiFigure} 
	\addFigure{0.4}{../Oral/Images/qm_mm.png}
	\addFigure{0.4}{../Oral/Images/qm_mm_pme.png}
	\captionof{figure}[QM/MM Diagram]{a) single cell. b) representation of the periodic nature of the system.}
	\label{fig:QMMMDiagram}
	\end{multiFigure}
\bigskip

	In the previous sections we have discussed how quantum mechanics can be used for chemical calculations.
  However, in many applications, the accuracy of QM is not needed and more computationally cheaper method would be more appropriate.
	For these situations many computational chemist use classical electrical force field dynamics, treating atoms as point charges.
	QM/MM was developed to manage computational costs by separating a calculation into a quantum mechanical (QM) region and a classical mechanical (MM) region.\cite{warshel1976theoretical,Karplus2014}
	This allows the user to have the accuracy where needed while not wasting resources on unwanted calculations such as the dynamics of water molecules far from the protein of interest.
	For the vast majority of our calculations, we will have a QM solute and a few nearby QM solvents surrounded by MM solvents.

	Figure \ref{fig:QMMMDiagram} gives an example of a QM/MM systems.
	The atoms of the drawn out molecule will be described at the QM level of theory.
	The MM atoms in the volume immediately surrounding the molecular, label QMCut, will be the MM atoms included in equation \ref{eq:qmmm}.
	To simulate a solute in solvent, we treat the provided box as a cell, that is repeated infinitely many times.
	Particle Mesh Ewald calculations are then used to calculate the long distance interactions of the periodic boxes.
	This is performed by treating the charge and potential in the long range, inter box distances, as sums in Fourier space.\cite{Darden1993}
	Note that the QM region must be treated as single point charges for this calculations.
	The Mulliken charges of the current state are used for these calculations.
	Once the sums are complete, a fast Fourier transform is performed to obtain energy and forces.
	Charges from the MM region outside QMCut, will be used to provide a Particle Mesh Ewald correction to the new Fock Matrix.\cite{Walker2008}

	Long range interaction, from those outside the cutoff, considered vital for the understanding of solvent effects, are treated using SQM’s implementation of Particle Mesh Ewald.
	Trajectories use periodic boundary conditions to simulate an explicit solution, treating the system box as cells repeated infinitely many times in all directions.
	Particle Mesh Ewald calculations then determine the long-distance interactions of these periodic boxes, treating the charges and potentials in the long-range inter-box distances as sums in Fourier space treating atoms in the QM region of these calculations as Mulliken point charges.
	Once the sums are complete, SQM performs a fast Fourier transformation to obtain the long-range corrections to the energy and forces.

\section{Organic Conjugated Molecules}
Conjugate organic polymers have been shown to exhibit ultra-fast exciton decay.\cite{nelson2018coherent,Fernandez-Alberti2009}
Studies have been performed to determine whether we can sythesize unidirectional energy transfers in these systems.\cite{soler2012analysis,soler2014signature,Galindo2015,FernandezAlberti2010,FernandezAlberti2012}
The have a dense manifold of electronic states.
They have strong electron-phonon couplings. (cite: sifain2018photoexcited 14-16)
They have photophysical properties that are rare (cite: sifain2018photoexcited 25,26)
Due also in part to their low cost of production a heavy interest has been show in using them for technological development. (cite: sifain2018photoexcited 17-24)
Experimentally, these molecules are studied either in solution or as solid state samples.
These types of scenarios have been too computationall expensive to siulate explicitely, and have only recently been studied using implice solvents. (cite: sifain and josiahs)

\section{Overview}
In chapter 2 we go into the theoretical methods employed to simulated the previously discossed processes.
In chapter 3 describe discuss the computation details in our implementationdescribe discuss the computation details in our implementation.
In chapter 4 we apply our methodology to investigate the steady state absorption and fluorescence experienced by PPV\(_3\)NO\(_2\) in various solvents.
We also investigate the change in behavior caused by including solvents in the QM region.
In chapter 5 we apply the non-adiabatic methodoly to analyze the effects included QM/MM solvents have the non-adiabatic relaxation of PPV\(_3\)NO\(_2\) 
Finally, in chapter 6 we summarize our findings and suggest possible routes for future work.
