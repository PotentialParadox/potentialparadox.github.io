\chapter{Introduction} \label{introduction}

\section{Prologue}

\section{Qualitative Overview of Non-Adiabatic Dynamics}

\subsection{Energy Transfer}

\subsection{Solvent Effects}

\subsection{QM/MM}

During ultra-fast photovolatic processes, the Born-Openheimer
appoximation breaks, and accounting for non-adiabatic behavior become
necessary.  These situations occur frequently within processes of
interest to photochemistry and photophysics.  For example, the
excitation to a non-equlibrium state followed by a relaxation through
internal conversion is a process common to processes such as
photosynthesis, solar-cell photo-absoprtion, optical detectors, and
the excitation of the visual nerve.  Multiple methods to have been
proposed and used to simulate these non-adiabatic processes.  These
methods include treating the nuclear coordinates quantum mechanically
or simiclassically, or by using a hybrid quantum mechanical, classical
treatment to account for the non-adiabaticity.  One of the more
popular version of the latter, and the one which we use in this work,
is Molecular Dynamics with Quantum Transitions (MDQT), were the system
propogates classically along adiabatic potential energy surfaces, but
a quantum evalutation is made at each time step to determine whether
to transition to another state.

For many areas in which nonadiabatic dynamics simulations would be of interest, solvents play a crucial role.
In situations where ultrafast electronic relaxations occur, the electronic decay is often faster than the time for the solvent to equilibrate.
As such, Implicit solvents, which adjusts instantaneously to any changes, become imprecise approximation.
However performing non-adiabatic dynamics on such large systems is far too computationally expensive.
To ease the computational cost we can employ QM/MM methodologies to perform the non-adiabatic calculation only on the areas of interest.
Similar methods have been employed in the study of retinal photochemistry and organic semiconductors.\cite{weingart2012modelling,demoulin2017fine,heck2015multi}%\cite{demoulin2017fine, weingart2012modelling, heck2015multi}
In this work we implement a new method of performing non-adiabatic QM/MM using the SANDER package AMBERTOOLS combined with the high performance Non-Adiabatic simulator NEXMD.
We further analyze the effects of including near solvent molecules within the QM region.
